% \chapter*{\centerline{АННОТАЦИЯ}}				% Заголовок
\chapter*{АННОТАЦИЯ}							% Заголовок

Дипломный проект \pageref*{LastPage} с., 5 ч., 36 рис., 11 табл., 18 источников.

\vspace{\baselineskip}
% ЭЛЕКТРОННЫЙ ДОКУМЕНТООБОРОТ, СЕТЬ МАССОВОГО ОБСЛУЖИВАНИЯ, АНАЛИЗ, ИМИТАЦИОННОЕ МОДЕЛИРОВАНИЕ, ЗАЩИЩЁННАЯ СИСТЕМА.
электронный документооборот, сеть массового обслуживания, имитационное моделирование, совместная работа, защищённая система

\vspace{\baselineskip}
В последнее время обретает актуальность проблема обеспечения информационной безопасности в системах электронного документооборота. 

Цель работы – создание средств обеспечения информационной безопасности для системы электронного документооборота, удовлетворяющих современным требованиям по быстродействию и защите информации.

В процессе работы проводился анализ существующих решений в данной области, а также основных угроз и средств противодействия им.

Система электронного документооборота была рассмотрена с точки зрения теории массового обслуживания. Для оценки её характеристик был првоедён ряд экспериментов с использованием метода имитационного моделирования.

В результате работы был осуществлён выбор методов обработки данных, а также создано программное обеспечение для защиты данных в такой системе.

Результаты в первую очередь значимы для специалистов, занимающихся созданием, внедрением и эксплуатацией систем электронного документооборота.

В дальнейшем на базе созданных средств путём разработки дополнительных модулей возможно создание полноценной системы электронного документооборота, обеспечивающей конкурентные функциональные возможности.

\clearpage