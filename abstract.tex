% \chapter*{\centerline{АННОТАЦИЯ}}				% Заголовок
\chapter*{\begin{center}РЕФЕРАТ\end{center}}							% Заголовок
Расчётно-пояснительная записка к дипломному проекту: \pageref*{LastPage} страниц, 5 частей, 36 рисунков, 11 таблиц, 18 источников, 1 приложение.

\vspace{\baselineskip}
\textit{Ключевые слова}: электронный документооборот, сеть массового обслуживания, имитационное моделирование, совместная работа, защищённая система

\vspace{\baselineskip}
В последнее время обретает актуальность проблема обеспечения информационной безопасности в системах электронного документооборота. 

Цель работы --- создание средств обеспечения информационной безопасности для системы электронного документооборота, удовлетворяющих современным требованиям по быстродействию и защите информации.

В процессе работы проводился анализ существующих решений в данной области, а также основных угроз и средств противодействия им.

Система электронного документооборота была рассмотрена с точки зрения теории массового обслуживания. Для оценки её характеристик был проведён ряд экспериментов с использованием метода имитационного моделирования.

В результате был осуществлён выбор методов обработки данных, а также создано программное обеспечение для защиты данных в такой системе.

В дальнейшем на базе созданных средств путём разработки дополнительных модулей возможно создание конкурентоспособной системы электронного документооборота.

Результаты работы значимы в первую очередь для специалистов, занимающихся созданием, внедрением и эксплуатацией систем электронного документооборота.

\clearpage