\section{Математическая постановка задачи} \label{math}

Пусть вероятность получения корректного документа (документа, не содержащего ошибок) после обработки i-м редактором в схеме рис. \ref{img:graph1} равна $p_i$. Тогда вероятность получения корректного документа после обработки последовательно $N$ редакторами равна
$$
P=\prod_{i=1}^N p_i.
$$
При обработке документов в соответствии со схемой рис. \ref{img:net} АС исправляет часть ошибок редактора. Детектируемые ошибки появляются при редактировании с вероятностью

\begin{equation}
  \label{eq:equation3}
P_{AC}=P_{data}+(1-P_{data})P_{auth}+(1-P_{data})(1-P_{auth})P_{den}+(1-P_{auth})(1-P_{data})(1-P_{den})p_{txt}.
\end{equation}

Вероятность отклонения автоматизированной системой (вероятность возврата на доработку) равна

\begin{equation}
  \label{eq:equation4}
P_{AC}'=P_{data}'+(1-P_{data}')P_{auth}'+(1-P_{data}')(1-P_{auth}')P_{den}'+(1-P_{auth}')(1-P_{data}')(1-P_{den}')P_{txt}',
\end{equation}
где вероятности $P_i'$ обозначают одновременное наступление двух событий: появление ошибки $k$ и её обнаружение соотв. модулем АС.

\vspace{\baselineskip}
Вероятность появления \textit{недетектируемых} ошибок -- $P_A$.

\vspace{\baselineskip}
Вероятность возникновения ошибки на $i$-том узле в процессе документооборота составляет $P_{A_i}+P_{{AC}_i}$. В случае использования АС для обнаружения ошибок она уменьшается до $P_{A_i}+(P_{{AC}_i}-P_{{AC}_i}')$.

\vspace{\baselineskip}
Пусть процесс документооборота характеризуется графом $G(V,E)$, изображённым на рис. \ref{img:graph2}.

\vspace{\baselineskip}
Пусть $M$ — множество всех возможных маршрутов при обработке документа. Тогда любой маршрут $m \in M$ характеризуется упорядоченным набором весов рёбер, лежащих на нём: $m=(p_{ij}); i,j \in V$. Тогда показатель эффективности СЭД соответствует вероятности получения корректного документа после обработки:
$$
\sum_{m \in M} (\prod_{p_{ij} \in m} p_{ij}(1-P_{err})),
$$
где $P_{{err}_j}=P_{A_j}+P_{{AC}_j}$ для классического документооборота и $P_{{err}_j}=P_{A_j}+(P_{{AC}_j}-P_{{AC}_j}')$ для СЭД.

\vspace{\baselineskip}
Коэффициент прироста показателя эффективности при применении вышеописанной схемы рассчитывается следующим образом:
$$
E_k=\frac{\sum_{m \in M} (\prod_{p_{ij} \in m} p_{ij}(1-(P_{A_j}+(P_{{AC}_j}-P_{{AC}_j}'))))}{\sum_{m \in M} (\prod_{p_{ij} \in m} p_{ij}(1-(P_{A_j}+P_{{AC}_j})))},
$$
где $P_{{AC}_j}$ рассчитывается по формуле (\ref{eq:equation3}), а $P_{{AC}_j}'$ -- по формуле (\ref{eq:equation4}).

\vspace{\baselineskip}
Задачей дипломного проектирования является повышение коэффициента прироста показателя  эффективности над заданным уровнем:
$$
\begin{cases}
   E_k=\frac{\sum_{m \in M} (\prod_{p_{ij} \in m} p_{ij}(1-(P_{A_j}+(P_{{AC}_j}-P_{{AC}_j}'))))}{\sum_{m \in M} (\prod_{p_{ij} \in m} p_{ij}(1-(P_{A_j}+P_{{AC}_j})))} > E_{lim} \\
   E_{lim} > 1
  \end{cases}
$$
\clearpage