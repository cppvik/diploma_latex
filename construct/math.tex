\section{Математическая постановка задачи} \label{math}

Пусть вероятность получения корректного документа (документа, не содержащего ошибок) после обработки i-м редактором в схеме рис. \ref{img:graph1} равна $p_i$. Тогда вероятность получения корректного документа после обработки последовательно $N$ редакторами равна
\begin{equation}
  \label{eq:correct_doc}
P=\prod_{i=1}^N p_i.
\end{equation}

При обработке документов в соответствии со схемой рис. \ref{img:net} АС исправляет часть ошибок редактора. Детектируемые ошибки появляются при редактировании с вероятностью

\begin{equation}
  \label{eq:equation3}
P_{AC}=P_{{auth}}+(1-P_{{auth}})P_{data}+(1-P_{{auth}})(1-P_{data})P_{txt}.
\end{equation}

Вероятность отклонения автоматизированной системой (вероятность возврата на доработку) равна

\begin{equation}
  \label{eq:equation4}
P_{AC}'=P_{{auth}}'+(1-P_{{auth}}')P_{data}'+(1-P_{{auth}}')(1-P_{data}')P_{txt}',
\end{equation}
где вероятности $P_i'$ обозначают одновременное наступление двух событий: появление ошибки $k$ и её обнаружение соответствующим модулем АС.

\vspace{\baselineskip}
Вероятность появления \textit{недетектируемых} ошибок --- $P_A$.

\vspace{\baselineskip}
Вероятность возникновения ошибки на $i$ \ndash том узле в процессе документооборота составляет $P_{A_i}\hm+P_{{AC}_i}$. В случае использования АС для обнаружения ошибок она уменьшается до $P_{A_i}\hm+(P_{{AC}_i}\hm-P_{{AC}_i}')$.

\vspace{\baselineskip}
Пусть процесс документооборота характеризуется графом $G(V,E)$, изображённым на рис. \ref{img:graph2}.

\vspace{\baselineskip}
Пусть $M = \{m_1, m_2, ... \}$ — множество всех возможных маршрутов при обработке документа. Тогда любой маршрут $m \in M$ характеризуется упорядоченным набором весов рёбер, лежащих на нём: $m=(p_{ij}); i,j \in V$. Тогда показатель эффективности СЭД соответствует вероятности получения корректного документа после обработки:
\begin{equation}
  \label{eq:mark}
\sum_{m \in M} (\prod_{p_{ij} \in m} p_{ij}(1-P_{\textrm{ош.}})),
\end{equation}
где $P_{{\textrm{ош.}}_j}\hm=P_{A_j}\hm+P_{{AC}_j}$ --- вероятность ошибки для классического документооборота. Для СЭД вероятность ошибки составляет $P_{{\textrm{ош.}}_j}\hm=P_{A_j}\hm+(P_{{AC}_j}\hm-P_{{AC}_j}')$.

% \vspace{\baselineskip}
% Коэффициент прироста показателя эффективности при применении вышеописанной схемы рассчитывается следующим образом:
% $$
% E_k=\frac{\sum_{m \in M} (\prod_{p_{ij} \in m} p_{ij}(1-(P_{A_j}+(P_{{AC}_j}-P_{{AC}_j}'))))}{\sum_{m \in M} (\prod_{p_{ij} \in m} p_{ij}(1-(P_{A_j}+P_{{AC}_j})))},
% $$
% где $P_{{AC}_j}$ рассчитывается по формуле (\ref{eq:equation3}), а $P_{{AC}_j}'$ --- по формуле (\ref{eq:equation4}).

\vspace{\baselineskip}
Задачей дипломного проектирования является повышение показателя эффективности:
\begin{equation}
  \label{eq:mathtask}
\begin{cases}
   E_k=\frac{\sum_{m \in M} (\prod_{p_{ij} \in m} p_{ij}(1-P_{\textrm{ош.}}')))}{\sum_{m \in M} (\prod_{p_{ij} \in m} p_{ij}(1-P_{\textrm{ош.}}))} = \frac{\sum_{m \in M} (\prod_{p_{ij} \in m} p_{ij}(1-(P_{A_j}+(P_{{AC}_j}-P_{{AC}_j}'))))}{\sum_{m \in M} (\prod_{p_{ij} \in m} p_{ij}(1-(P_{A_j}+P_{{AC}_j})))} > E_{\textrm{пред.}} \\
   E_{\textrm{пред.}} > 1
  \end{cases}
\end{equation}
% \clearpage