\section{Расчёт трудоёмкости проекта} \label{economics_laboriousness}
% \addcontentsline{toc}{chapter}{Понятие системы электронного документооборота}  % Добавляем его в оглавление

Общие затраты труда на разработку и ПО определим следующим образом:
\begin{equation}
  \label{eq:common_laboriousness}
Q_p = \sum_{i} T_i,
\end{equation}
где $T_i$ -- затраты труда на выполнение $i$-го этапа проекта.

Используя метод экспертных оценок, вычислим ожидаемую продолжительность работ $T$ каждого этапа по формуле:
\begin{equation}
  \label{eq:work_duration}
T = \frac {3 \cdot T_{MIN} + 2 \cdot T_{MAX}} {5},
\end{equation}
где $T_{MAX}$ и $T_{MIN}$ – максимальная и минимальная продолжительность работы. Они назначаются в соответствии с экспертными оценками, а ожидаемая продолжительность работы рассчитывается как математическое ожидание для $\beta$- распределения.

\vspace{\baselineskip}
Полный перечень работ с разделением их по этапам приведён в таблице \ref{table:works}.
\begin{table} [h!]
  \centering
  \parbox{15cm}{\caption{Распределение работ по этапам}\label{table:works}}
 \begin{center}
  \begin{tabular}{| c | m{3cm} | c | m{5cm} | m{1cm} | m{1cm} | m{1cm} | m{1cm} |}
  \hline
 \rowcolor{Gray} №  &\centering Этап &\centering № работы &\centering  Содержание работы &\centering $T_{MIN}$, чел / часы &\centering $T_{MAX}$, чел / часы &\centering $T$, чел / часы &\centering $T$, чел / дни \tabularnewline \hline

\multirow{4}{*}{1} 	& \multirow{4}{3cm}{Разработка технических требований}	& 1 & Получение задания, анализ полученных требований к разрабатываемому ПО		& 8 & 8	& 8	& 1 \\ \cline{3-8}
 	& & 2 & Разработка и утверждение ТЗ 	& 24 & 24 & 24 & 3 \\ \cline{3-8}
 	& & 3 & Анализ предметной области и существующих решений & 24 & 44 & 32  & 4 \\ \cline{3-8}
 	& & 4 & Анализ потоков данных в процессе электронного документооборота & 72 & 92 & 80 & 10 \\ \hline

\multirow{2}{*}{2} & \multirow{2}{3cm}{Разработка алгоритмов} & 5 & Разработка общей структуры ПО и пользовательского интерфейса & 24 & 44 & 32 & 4 \\ \cline{3-8}
	& & 6 & Разработка алгоритмов, структуры входных и выходных данных & 64 & 84 & 72 & 9 \\ \hline

\multirow{2}{*}{3} & \multirow{2}{3cm}{Разработка программных модулей} & 7 & Реализация пользовательского интерфейса & 32 & 52 & 40 & 5 \\ \cline{3-8}
	& & 8 & Программная реализация модулей защищенной обработки, передачи и хранения информации & 72 & 92 & 80 & 10 \\ \hline

\multirow{4}{*}{4} & \multirow{4}{3cm}{Тестирование и отладка разрабатываемого ПО} & 9 & Тестирование ПО & 64 & 84 & 72 & 9 \\ \cline{3-8}
	& & & & & & & \\
	& & 10 & Внесение изменений в ПО & 32 & 52 & 40 & 5 \\
	& & & & & & & \\ \hline

5 & Разработка документации & 11 & Разработка программной и эксплуатационной документации & 64 & 84 & 72 & 9 \\ \hline

\multicolumn{4}{|l|}{Итого $Q_P$:} & \multicolumn{3}{|l|}{616} & 77 \\ \hline
   \end{tabular}
 \end{center}
\end{table}

 $$
Q_P = Q_{\textrm{ОЖ}} = 77 (\textrm{чел/дней}) = 616 (\textrm{чел/час}).
 $$