\section{Построение сетевого графика} \label{net_graph}

Для определения временных затрат и трудоемкости разработки ПО используем метод сетевого планирования. Метод сетевого планирования позволяет установить единой схемой связь между всеми работами в виде наглядного и удобного для восприятия изображения (сетевого графика), представляющего собой информационно-динамическую модель, позволяющую определить продолжительность и трудоёмкость, как отдельных этапов, так и всего комплекса работ в целом.

\vspace{\baselineskip}
Составление сетевой модели включает в себя оценку степени детализации комплекса работ и определения логической связи между отдельными работами.
С этой целью составляется перечень всех основных событий и работ. В перечне указываются кодовые номера событий, наименования событий в последовательности от исходного к завершающему, кодовые номера работ, перечень всех работ, причём подряд указываются все работы, которые начинаются после наступления данного события.

\vspace{\baselineskip}
Основные события и работы проекта представлены в таблице \ref{table:events}.

\vspace{\baselineskip}
Рассчитанные оставшиеся параметры элементов сети (сроки наступления событий, резервы времени событий, полный и свободный резервы времени работ) приведены в таблице \ref{table:time_per_work}.