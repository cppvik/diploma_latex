\subsection{Затраты на организацию рабочих мест} \label{office_costs}

Расчет затрат, связанных с организацией рабочих мест для исполнителей проекта, проводится на основе требований СНИПа (санитарные нормы и правила) и стоимости аренды помещения требуемого уровня сервиса.

\vspace{\baselineskip}
В соответствии с санитарными нормами расстояние между рабочими столами с видеомониторами должно быть не менее 2 м., а между боковыми поверхностями видеомониторов - не менее 1,2 м. Площадь на одно рабочее место с терминалом или ПК должна составлять не менее 6 кв.м., а объем - не менее 20 куб.м.. Расположение рабочих мест в подвальных помещениях не допускается. Помещения должны быть оборудованы системами отопления, кондиционирования воздуха или эффективной приточно-вытяжной вентиляцией. Таким образом, для размещения двух сотрудников и принтера необходимо помещение (комната) площадью 6+6+3=15 кв. м.

\vspace{\baselineskip}
Затраты на аренду помещения можно вычислить исходя из следующего соотношения:
\begin{equation}
  \label{eq:office_cost_formula}
C_\textrm{ОРГ} = \frac {C_\textrm{КВМ}} {12} \cdot S \cdot T_{AP},
\end{equation}
где $C_\textrm{КВМ}$ --- стоимость аренды одного квадратного метра площади за год, $S$ --- арендуемая площадь рабочего помещения, $T_{AP}$ --- срок аренды (мес).

\vspace{\baselineskip}
В настоящее время возможна аренда не офисного помещения, а раздельных рабочих мест, оборудованных всеми необходимыми коммуникациями, мебелью и оргтехникой. Стоимость обслуживания каналов телекоммуникации, а также расходных материалов для оргтехники включена в стоимость аренды рабочих мест. Для бизнец-центра <<Matrixoffice>> (м. Шаболовская) стоимость аренды одного рабочего места составляет 9 000 рублей. С учётом относительно небольших сроков разработки проекта и небольшого штата сотрудников, целесобразно арендовать не отдельное офисное помещение, а рабочие места.
Таким образом, стоимость аренды составляет
\begin{equation}
  \label{eq:office_cost}
C_\textrm{ОРГ} = 18 000 \cdot 3 = 54000 (\textrm{руб.})
\end{equation}