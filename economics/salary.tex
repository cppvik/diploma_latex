\subsection{Затраты на выплату заработной платы} \label{salary}

Затраты на выплату исполнителям заработной платы линейно связаны с трудоемкостью и определяется следующим соотношением:
\begin{equation}
  \label{eq:salary}
C_\textrm{ЗАРП} = C_\textrm{З.ОСН} + C_\textrm{З.ДОП} + C_\textrm{З.ОТЧ},
\end{equation}
где $C_\textrm{З.ОСН}$ --- основная заработная плата, $C_\textrm{З.ДОП}$ --- дополнительная заработная плата, $C_\textrm{З.ОТЧ}$ --- отчисление с заработной платы.

\vspace{\baselineskip}
Расчёт основной заработной платы (оплаты труда непосредственных исполнителей):
\begin{equation}
  \label{eq:salary_counting}
C_\textrm{З.ОСН} = T_\textrm{ЗАН} \times O_\textrm{ДН},
\end{equation}
где $T_\textrm{ЗАН}$ --- число дней, отработанных исполнителем проекта, а $O_\textrm{ДН}$ --- днейвной оклад исполнителя, который при 8-часовом рабочем дней рассчитывается по формуле:
\begin{equation}
  \label{eq:worker_fee}
O_\textrm{ДН} = \frac {O_\textrm{МЕС} \cdot 8} {F_M},
\end{equation}
где $O_\textrm{МЕС}$ --- месячный оклад, а $F_M$ --- месячный фонд рабочего времени (\ref{eq:time_fund_month_common}).

\vspace{\baselineskip}
С учетом налога на доходы физических лиц размер оклада увеличивается:
\begin{equation}
  \label{eq:worker_fee_with_taxes}
O_\textrm{МЕС} = O \cdot (1 + \frac {H_\textrm{ДФЛ}} {100}),
\end{equation}
где $O$ --- <<чистый>> оклад, $H_\textrm{ДФЛ}$ --- налог на доходы физических лиц ($13\%$).

\vspace{\baselineskip}
Сведем результаты расчета в таблицу с перечнем исполнителей и их месячных и дневных окладов, а также времени участия в проекте и рассчитанной основной заработной платой каждого исполнителя (таблица \ref{table:fee_all_workers}).

\begin{table} [h!]
  \captionsetup{justification=raggedright}
  \caption{Заработная плата исполнителей}\label{table:fee_all_workers}
 \begin{center}
  \begin{tabular}{| c | >{\centering}m{3cm} | >{\centering}m{2.5cm} | >{\centering}m{2.5cm} | >{\centering}m{2.5cm} | >{\centering}m{2.5cm} |}
  \hline
 \rowcolor{Gray} №  & Должность & <<Чистый>> оклад, руб. & Дневной оклад, руб. &  Трудозатраты, чел-дни & Затраты на зарплату, руб. \tabularnewline \hline

 1 & Ведущий программист & 60 000 & 3267,47 & 54 & 176443,37 \tabularnewline \hline
 2 & Программист & 50 000 & 2722,89 & 15 & 40843,37 \tabularnewline \hline

   \end{tabular}
 \end{center}
\end{table}

Из таблицы получим общие затраты проекта на заработную плату исполнителей: $C_\textrm{З.ОСН} = 217286,24$ руб.

\vspace{\baselineskip}
Расходы на дополнительную заработную плату учитывают все выплаты непосредственным исполнителям за время, не проработанное производстве, но предусмотренное законодательством. Величина выплат составляет $20\%$ от размера основной заработной платы: 
\begin{equation}
  \label{eq:worker_extra_fee}
C_\textrm{З.ДОП} = 0.2 \cdot C_\textrm{З.ОСН} = 0.2 \cdot 217286,24 = 43457,25 (\textrm{руб.})
\end{equation}
\FloatBarrier