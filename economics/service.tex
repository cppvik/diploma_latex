\subsection{Сервисное обслуживание} \label{service}

Сервисное обслуживание программного обеспечения будет выполнять один сотрудник. Для того, чтобы не обучать новый персонал особенностям нашего программного обеспечения для выполнения данной работы привлечем программиста, который участвовал в разработке программного обеспечения. Так как мы продаём по 5 экземпляров нашего продукта в месяц, то время, которое затратит программист, составит 5 рабочих дней в месяц, 60 дней за год. Затраты на сервисное обслуживание приведены в таблице \ref{table:economics_service}

\begin{table} [h!]
  \captionsetup{justification=raggedright}
  \caption{Заработная плата исполнителей}\label{table:economics_service}
 \begin{center}
  \begin{tabular}{| c | >{\centering}m{3cm} | >{\centering}m{2cm} | >{\centering}m{2cm} | >{\centering}m{2.5cm} | >{\centering}m{2cm} |}
  \hline
 \rowcolor{Gray} № & Должность & <<Чистый>> оклад, руб. & Дневной оклад, руб. & Трудозатраты, чел–дни & Затраты на зарплату, руб. \tabularnewline \hline

1 & Программист & 50 000 & 2722,89 & 60 & 163373,40 \tabularnewline \hline

\end{tabular}
\end{center}
\end{table}

Расходы на дополнительную заработную плату учитывают все выплаты непосредственным исполнителям за время, не проработанное на производстве, но предусмотренное законодательством. Величина этих выплат составляет 20\% от размера основной заработной платы:
\begin{equation}
  \label{eq:service_extra_fee}
C_\textrm{З.ДОП} = 0.2 \cdot C_\textrm{З.ОСН} = 0.2 \cdot 163373,40 = 32674,68 (\textrm{руб.})
\end{equation}
\FloatBarrier