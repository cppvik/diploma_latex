\subsection{Определение численности исполнителей} \label{workers}

Для оценки возможности выполнения проекта имеющимся в распоряжении разработчика штатным составом исполнителей нужно рассчитать их среднее количество, которое при реализации проекта разработки и внедрения ПО определяется соотношением:
\begin{equation}
  \label{eq:common_workers}
N = \frac {Q_P} {F},
\end{equation}
где $Q_P$ -- затраты труда на выполнение проекта (разработка и внедрение ПО), а $F$ -- фонд рабочего времени, который определяется по формуле:
\begin{equation}
  \label{eq:time_fund_month_common}
F_M = T \cdot \frac {t_P \cdot (D_K - D_B - D_\textrm{П})} {12},
\end{equation}
где $T$ -- время выполнения проекта в месяцах, $t_P$ -- продолжительность рабочего дня, $D_K$ -- общее число дней в году, $D_B$ -- число выходных дней в году, $D_\textrm{П}$ -- число праздничных дней в году.

\vspace{\baselineskip}
Таким образом, фонд времени в текущем месяце 2014 года составляет
\begin{equation}
  \label{eq:time_fund_month}
F_M = \frac {8 \cdot (365 - 104 - 14)} {12} = 165 \textrm{ часов/мес}.
\end{equation}

Время выполнения проекта $T = 3,5$ (месяца).

\vspace{\baselineskip}
Величина фонда рабочего времени составляет:
\begin{equation}
  \label{eq:time_fund}
F = T \cdot F_M = 577,5 \textrm{ ч}.
\end{equation}

Затраты труда на выполнения проекта были рассчитаны в предыдущем разделе, их величина равна $616 \textrm{ чел/час}$. В соответствии с этими данными и выражением (\ref{eq:common_workers}), среднее количество исполнителей равно:
\begin{equation}
  \label{eq:workers_avg}
N = \frac {616} {577,5} = 1,07.
\end{equation}

Округляя до большего, получим число исполнителей проекта $N = 2$. 