\chapter*{ВВЕДЕНИЕ}							% Заголовок
\addcontentsline{toc}{chapter}{ВВЕДЕНИЕ}	% Добавляем его в оглавление

\textbf{\textit{Документооборот}} --- движение документов в организации с момента их создания или получения до завершения исполнения или отправления \cite{gost51141}.
Документооборот является неотъемлемой частью рабочего процесса любой компании любого масштаба. При выполнении производственных задач, организации процессов внутрии компании, коммуникации с контрагентами и органами государственной власти используются документы. Правильная организация документооборота способна повысить эффективнсть работы предприятия и оптимизировать временн\'{ы}е и материальные затраты.

\vspace{\baselineskip}
В течение последнего десятилетия наблюдается постепенный переход от бумажного документооборота к электронному. Это может проявляться как в полном или частичном отказе от бумажных версий документов, так и в дублировании бумажных копий электронными. Положительные моменты этого процесса --- такие, как повышение скорости обработки документов и снижение материально-временных затрат на создание, хранение и передачу документов --- компенсируются сложностями в обеспечении информационной безопасности электронных документов. Так, необходимо обеспечить защиту от несанкционированного доступа к хранилищу документов и каналу передачи данных, а также создать средства подтверждения авторства документа. Данный вопрос лежит как в технической, так и в правовой области.

\vspace{\baselineskip}
Очевидна необходимость в создании системы электронного документооборота, сохраняющей описанные достоинства и минимизирующей недостатки, т.е. такой системы, которая будет обеспечивать как осуществление процесса электронного документооборота, так и безопасность обрабатываемых данных.

\vspace{\baselineskip}
Основными задачами дипломного проектирования являются:
% \begin{itemize}
% 	\item проведение анализа угроз и средств противодействия им для систем электронного документооборота;
% 	\item осуществление выбора показателей эффективности реализации модулей системы электронного документооборота;
% 	\item выбор методов обработки данных в системе электронного документооборота;
% 	\item разработка программного обеспечения для защиты данных в системе электронного документооборота;
% 	\item проведение анализа нормативно-правовых актов в области информационной безопасности Российской Федерации;
% 	\item проведение организационно-экономического анализа проектной разработки, оценка структуры и показателей затрат дипломного проекта, исследования рынка.
% \end{itemize}
\begin{itemize}
	\item анализ существующих систем защиты информации в системах электронного документооборота;
	\item выбор показателей эффективности реализации модулей системы электронного документооборота;
	\item выбор методов обработки данных в системе электронного документооборота;
	\item разработка программного обеспечения для защиты данных в системе электронного документооборота;
	\item анализ нормативно-правовых актов в области информационной безопасности Российской Федерации;
	\item организационно-экономический анализ проектной разработки, оценка структуры и показателей затрат дипломного проекта, исследования рынка.
\end{itemize}

\clearpage