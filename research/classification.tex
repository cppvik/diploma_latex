% \chapter{Классификация систем электронного документооборота} \label{chapt3}
\subsection{Классификация систем электронного документооборота} \label{review_classification}

По классу решаемых задач выделяют СЭД:
\begin{itemize}
	\item Ориентированные на бизнес-процессы

	Развитое управление и процессами, и содержимым в контексте отрасли.
	\item Корпоративные

	Общекорпоративные системы.
	\item Системы управления содержимым

	Объектом является документ.
	\item Системы управления потоком работ

	Объектом является работа.
	\item Системы управления образами

	Большое внимание уделяется вводу документов из бумажных форм (в виде отсканированных изображений) и перевод их в электронный вид.
	\item Системы управления корпоративными электронными записями

	Работа с неизменяемыми данными (например, денежными транзакциями).
\end{itemize}

\vspace{\baselineskip}
По споcобу распространения СЭД делятся на
\begin{itemize}
	\item Коробочные

	Универсальные системы, которые потребитель может настроить для своих нужд самостоятельно.
	\item Проектные

	Системы <<под заказ>>, разворачиваются и адаптируются индивидуально для предприятия.
\end{itemize}
% \clearpage