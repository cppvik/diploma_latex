% \chapter{Назначение и основные свойства систем электронного документооборота} \label{chapt2}
\subsection{Назначение и основные свойства систем электронного документооборота} \label{review_common}

Документооборот включает в себя:
\begin{itemize}
	\item создание
	\item обработку
	\item хранение
	\item передачу
	\item вывод документов.
\end{itemize}

\vspace{\baselineskip}
Соответственно, в задачи системы документооборота входит обеспечение этих процессов, причём в разных системах упор делается на разные стадии. Это происходит за счёт интеграции со сторонними средствами – например, средствами сканирования и распознавания текста.

\vspace{\baselineskip}
К основным свойствам СЭД относят:
\begin{itemize}
	\item Открытость;
		
		СЭД строятся по модульному принципу, что позволяет подстраиваться под требования к системе, совершенствовать отдельные модули и интегрироваться со сторонними модулями.
	\item Высокая степень интеграции с прикладным программным обеспечением;

	Снижает затраты на обучение сотрудников: последние работают с привычным ПО, которое, в свою очерeдь, взаимодействует с СЭД.
	\item Организация хранения документов;

	К этому вопросу можно подойти разносторонне, но в основном выделяют три реальных модуля хранения данных, взаимодействующих между собой:
	\begin{itemize}
		\item Хранилище документов;
		\item Хранилище атрибутов документов;
		\item Сервисы индексации и поиска.
	\end{itemize}
\end{itemize}

\vspace{\baselineskip}
На их основе можно строить виртуальные сущности вроде работы с привязанными к ней документами.
\begin{itemize}
	\item Организация маршрутизации документов;

	В зависимости от решаемых задач, маршрутизация может быть свободной (меняться по мере работы с документом) или жёсткой (задаваться при создании задачи, без права исполнителя изменить маршрут).
	\item Разграничение доступа;

	Один из основных механизмов, обеспечивающих безопасность обрабатываемой информации – контроль доступа. Основные виды полномочий:
	\begin{itemize}
		\item Полный контроль документа;
		\item Разрешение на редактирование;
		\item Разрешение на создание новых версий;
		\item Право на чтение;
		\item Право на доступ к учётной карточке, без разрешения на редактирование документа;
		\item Право на доступ к карточке без доступа к документу;
		\item Полный запрет на работу с документом.
	\end{itemize}
	В зависимости от конкретной СЭД набор параметров может различаться.
	\item Поддержка версионности документов;

	Важное свойство для хранения достоверной истории и подтверждения авторства пользовательских изменений.
	\item Поддержка различных форматов данных;

	В зависимости от компании и решаемой задачи формат рабочего документа может варьироваться от ODT/DOCX до TeX и даже CAD. Выбор СЭД обуславливается в частности и поддержкой нужных форматов.
	\item Аннотирование документов.

	Полезное свойство для гибкости разграничения доступа: в некоторых ситуациях доступ к редактированию документа может быть излишним, но возможность добавления комментариев решает эту проблему.
\end{itemize}

% \clearpage