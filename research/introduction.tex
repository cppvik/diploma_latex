% \chapter{Почему электронный документооборот важен} \label{chapt1}
\section{Введение} \label{subsect1_1}
% \addcontentsline{toc}{chapter}{Понятие системы электронного документооборота}  % Добавляем его в оглавление

\textbf{\textit{Документооборот}} -- движение документов в организации с момента их создания или получения до завершения исполнения или отправления \cite{bib1}.
Документооборот является неотъемлемой частью рабочего процесса любой компании любого масштаба. При выполнении производственных задач, организации процессов внутрии компании, коммуникации с контрагентами и органами государственной власти используются документы. Правильная организация документооборота способна повысить эффективнсть работы предприятия и оптимизировать временные и материальные затраты.

\vspace{\baselineskip}
В течение последнего десятилетия наблюдается постепенный переход от бумажного документооборота к электронному. Это может проявляться как в полном или частичном отказе от бумажных версий документов, так и в дублировании бумажных копий электронными. Положительные моменты этого процесса -- такие, как повышение скорости обработки документов и снижение материально-временных затрат на создание, хранение и передачу документов -- компенсируются сложностями в обеспечении информационной безопасности электронных документов. Так, необходимо обеспечить защиту от несанкционированного доступа к хранилищу документов и каналу передачи данных, создать средства подтверждения авторства документа. Данный вопрос лежит как в технической, так и в правовой области.

\vspace{\baselineskip}
Для организации процесса электронного документооборота создаются программные и программно-аппаратные комплексы. Для решения задач информационной безопасности они так или иначе используют средства криптографической защиты информации. Однако среди используемых СКЗИ встречаются в основном средства, основанные на технологии Microsoft CryptoAPI -- криптопровайдеры, совместимые только с ограниченным наобором операционных систем Microsoft Windows. Более того, эти средства применяются только для решения двух наиболее очевидных задач: обеспечение работы с электронной подписью в соответстии с Федеральным законом №63 <<Об электронной подписи>> и защита канала передачи данных между хранилищем электронных документов и операторами системы. Без внимания остаются другие вопросы безопасности -- такие, как, например, обеспечение целостности хранимой истории изменений, внесённых в документ.

\vspace{\baselineskip}
В то же время, открытый подход к разработке программного обеспечения (когда любой желающий может просмотреть исходные тексты и предложить свои изменения) доказал свою состоятельность: к примеру, операционные системы, разработанные таким образом, используются в Роскосмосе. Плюсы данного подхода очевидны: за счёт открытости исходных текстов для так называемых <<белых хакеров>> и программистов по всему миру в итоговом ПО присутствует меньше ошибок по сравнению с аналогами. Разработанное таким способом ПО после внесения некоторых правок может быть сертифицировано и распространяться в соответствии с законодательством РФ.

\vspace{\baselineskip}
Таким образом, целью дипломного проектирования является анализ угроз и средств противодействия им для СЭД, выбор методов и средств защиты при передаче и хранении информации в СЭД, а также их реализация.