\section{Введение} \label{research_introduction}
% \addcontentsline{toc}{chapter}{Понятие системы электронного документооборота}  % Добавляем его в оглавление

% \textbf{\textit{Документооборот}} --- движение документов в организации с момента их создания или получения до завершения исполнения или отправления \cite{gost51141}.
% Документооборот является неотъемлемой частью рабочего процесса любой компании любого масштаба. При выполнении производственных задач, организации процессов внутрии компании, коммуникации с контрагентами и органами государственной власти используются документы. Правильная организация документооборота способна повысить эффективнсть работы предприятия и оптимизировать временные и материальные затраты.

% \vspace{\baselineskip}
% В течение последнего десятилетия наблюдается постепенный переход от бумажного документооборота к электронному. Это может проявляться как в полном или частичном отказе от бумажных версий документов, так и в дублировании бумажных копий электронными. Положительные моменты этого процесса --- такие, как повышение скорости обработки документов и снижение материально-временных затрат на создание, хранение и передачу документов --- компенсируются сложностями в обеспечении информационной безопасности электронных документов. Так, необходимо обеспечить защиту от несанкционированного доступа к хранилищу документов и каналу передачи данных, создать средства подтверждения авторства документа. Данный вопрос лежит как в технической, так и в правовой области.

\vspace{\baselineskip}
Для организации процесса электронного документооборота создаются программные и программно\ndashаппаратные комплексы. Для решения задач информационной безопасности они так или иначе используют средства криптографической защиты информации. Однако среди используемых СКЗИ встречаются в основном средства, основанные на технологии Microsoft CryptoAPI --- криптопровайдеры, совместимые только с ограниченным наобором операционных систем Microsoft Windows. Более того, эти средства применяются только для решения двух наиболее очевидных задач: обеспечения работы с электронной подписью в соответстии с Федеральным законом №63 <<Об электронной подписи>> и защиты канала передачи данных между хранилищем электронных документов и операторами системы. Без внимания остаются другие вопросы безопасности --- такие, как, например, обеспечение целостности хранимой истории изменений, внесённых в документ.

\vspace{\baselineskip}
В то же время, открытый подход к разработке программного обеспечения (когда любой желающий может просмотреть исходные тексты и предложить свои изменения) доказал свою состоятельность: к примеру, операционные системы, разработанные таким образом, используются в Роскосмосе \cite{roskosmos}. Плюсы данного подхода очевидны: за счёт открытости исходных текстов для так называемых <<белых хакеров>> и программистов по всему миру в итоговом ПО присутствует меньше ошибок по сравнению с аналогами.
% Разработанное таким способом программное обеспечение после внесения некоторых правок может быть сертифицировано и распространяться в соответствии с законодательством РФ.
% Системы электронного документооборота не подлежат обязательной сертификации, а значит, для их использования достаточно лишь задействовать сертифицированные СКЗИ.

\vspace{\baselineskip}
Для системы электронного документооборота немаловажным является фактор скорости передачи и обработки данных, дающий СЭД преимущество перед классическим (бумажным) документооборотом.

\vspace{\baselineskip}
Система электронного документооборота может быть разделена на модули, каждый из которых выполняет ту или иную операцию. Для каждого модуля исходя из особенностей его работы можно оценить набор угроз и в соответствии с этими угрозами определить состав средств противодействия им.

\vspace{\baselineskip}
При создании СЭД важно учитывать, что в системе обрабатываются данные, составляющие коммерческую, служебную, производственную и другие виды тайн. Важно обеспечить защиту этих данных как в канале передачи данных, так и в локальном хранилище пользователя и сервера.

\vspace{\baselineskip}
Таким образом, \textbf{целью дипломного проектирования является создание средств обеспечения информационной безопасности для системы электронного документооборота, удовлетворяющих современным требованиям по быстродействию и защите информации}.