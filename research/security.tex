% \chapter{Вопросы информационной безопасности} \label{chapt4}
\subsection{Вопросы информационной безопасности} \label{review_security}

В любой системе электронного документооборота должны присутствовать базовые функции безопасности:
\begin{itemize}
	\item Идентификация и аутентификация пользователей;

	Иначе применение СЭД бессмысленно
	\item Защита каналов передачи данных;

	Для противодействия атакам типа «человек посередине»
	\item Поддержка средств добавления и проверки электронной подписи;

	Для обеспечения юридической значимости обрабатываемым документам
	\item Строгое журналирование;

	Вместе с системой идентификации/аутентификации должно обеспечиваться жёсткое связывание автора и сделанных им изменений. При этом отмена изменений должна быть новой записью в журнале, а не отменой предыдущей.
	\item Резервирование сервисов (серверных средств), в т.ч. горячее.

	Для обеспечения свойства доступности
\end{itemize}

\vspace{\baselineskip}
Помимо этого, если в системе обрабатываются данные, защита которых предусмотрена действующим законодательством, должны выполнятсья и соответствующие требования.
% \clearpage