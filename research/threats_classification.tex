\subsection{Классификация угроз информационной безопасности систем электронного документооборота} \label{threats_classification}

Угрозы СЭД можно сгруппировать по нарушаемым свойствам безопасности: 
\begin{itemize}
	\item угрозы конфиденциальности:
	\begin{itemize}
		\item кража;
		\item перехват информации;
		\item изменение маршрутов следования информации.
	\end{itemize}
	\item угрозы целостности --- угрозы, при реализации которых информация теряет заранее определенные системой вид и качество. Объектами данной угрозы могут быть все компоненты СЭД:
	\begin{itemize}
		\item документы;
		\item резервные копии документов;
		\item среда хранения электронных документов;
		\item операционные системы и компоненты СЭД, установленные на клиентских рабочих станциях;
		\item каналы связи.
	\end{itemize}
	\item угрозы доступности, характеризующие возможность доступа к хранимой и обрабатываемой в СЭД информации в любой момент времени.
\end{itemize}

\vspace{\baselineskip}
В табл. \ref{table:threats} представлены возможные воздействия на элементы СЭД, приводящие к нарушению их нормального функционирования и, как к одному из сдледствий, потере юридической значимости электронных документов \cite{threats}.
 \begin{center}
 \renewcommand\multirowsetup{\centering}
 \begin{longtable}[h]{| >{\centering}m{5cm} | >{\centering}m{5cm} | >{\centering}m{5cm} |}
	
	\caption{Угрозы безопасности информации в СЭД} \label{table:threats} \tabularnewline
	\hline

 \rowcolor{Gray}   Элемент СЭД & Вид деструктивного воздействия &  Результат воздействия \tabularnewline \hline \endfirsthead   \hline
 \multicolumn{3}{|c|}{\small\slshape (продолжение)}        \tabularnewline \hline
 \rowcolor{Gray}   Элемент СЭД & Вид деструктивного воздействия &  Результат воздействия \tabularnewline \hline
                                              \endhead        \hline
 \multicolumn{3}{|r|}{\small\slshape продолжение следует}  \tabularnewline \hline
                                              \endfoot        \hline
                                              \endlastfoot

 \multirow{5}{5cm}{Система хранения и обработки документов} & Воздействие на носитель информации & Нарушение целостности и доступности ЭД, истории и метаданных \tabularnewline \cline{2-3}
 		& Внесение искажений в подписанный электронной подписью ЭД & Нарушение целостности ЭД \tabularnewline \cline{2-3}
 		& Несанкционированное изменение метаданных документа (формат, реквизиты, и т.п.) & Нарушение целостности, возможное нарушение доступности ЭД \tabularnewline \cline{2-3}
 		& Несанкционированное внесение изменений в историю ЭД & Нарушение целостности, возможное нарушение доступности ЭД \tabularnewline \cline{2-3}
 		& Нелегитимное копирование ЭД & Нарушение конфиденциальности ЭД и истории \tabularnewline \hline

 \multirow{2}{5cm}{Система передачи информации} & DoS каналов связи, отказ комплектующих & Нарушение целостности и доступности ЭД. Нарушение доступности CRL и сервера доверенного времени \tabularnewline \cline{2-3}
 		& Внесение искажений в передаваемую информацию & Нарушение целостности ЭД \tabularnewline \hline

 \multirow{2}{5cm}{Система разграничения доступа} & Компрометация аутентификационных данных легитимного пользователя & Нарушение конфиденциальности \tabularnewline \cline{2-3}
 		& Ошибки администрирования (преднамеренные и непреднамеренные): разрешение на доступ нелегитимных пользователей, запрет доступа легитимным пользователям & Нарушение  конфиденциальности. Нарушение доступности \tabularnewline \hline

 \multirow{4}{5cm}{Система проверки подлинности} & Компрометация ключа ЭП либо ключа удостоверяющего центра &  \multirow{2}{5cm}{Нарушение целостности ЭД} \tabularnewline \cline{2-2}
 		& Нелигитимное делигирование уполномоченным лицом права подписи ЭД & \tabularnewline \cline{2-3}
 		& Компрометация списка отозванных сертификатов & Нарушение достоверности ЭП \tabularnewline \cline{2-3}
 		& Отказ носителя ключа ЭП & нарушение доступности ключа ЭП, невозможность штатной работы СЭД \tabularnewline \hline
 
\end{longtable}
\end{center}

\vspace{\baselineskip}
Защиту от этих угроз в той или иной мере должна реализовывать любая система электронного документооборота. При этом, с одной стороны, при внедрении СЭД увеличиваются риски реализации угроз, но, с другой стороны, при правильном подходе упорядочение документооборота позволяет выстроить более качественную систему защиты.

\vspace{\baselineskip}
Таким образом, любая защищенная СЭД должна предусматривать реализацию как минимум следующих механизмов защиты: 
\begin{itemize}
	\item обеспечение целостности документов;
	\item обеспечение безопасного доступа;
	\item обеспечение конфиденциальности документов;
	\item обеспечение подлинности документов;
	\item протоколирование действий пользователей.
\end{itemize}

