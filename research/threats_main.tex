\section{Анализ угроз и средств противодействия им для систем электронного документооборота} \label{threats_main}

Считается, что для защиты СЭД достаточно использования электронной подписи (ЭП), однако в большинстве случаев разработчики не поясняют, как правильно использовать ЭП, какая нужна инфраструктура и какие защищённые сервисы необходимо развернуть на её основе \cite{buldakova}. Обычно ими приводятся только конкретные примеры реализованных защищенных СЭД. Поэтому понятие защищённого электронного документооборота (ЗЭД) определить достаточно трудно, особенно в условиях активно развивающихся технологий. В общем случае к задаче создания такой системы необходимо подходить с точки зрения классической защиты информационной системы, обеспечивая решение таких задач, как:
\begin{itemize}
	\item аутентификация пользователей и разделение доступа; 
	\item подтверждение авторства электронного документа; 
	\item контроль целостности электронного документа; 
	\item конфиденциальность электронного документа; 
	\item обеспечение юридической значимости электронного документа.
\end{itemize}

\subsection{Классификация угроз информационной безопасности систем электронного документооборота} \label{threats_classification}

Угрозы СЭД можно сгруппировать по нарушаемым свойствам безопасности: 
\begin{itemize}
	\item угрозы конфиденциальности:
	\begin{itemize}
		\item кража;
		\item перехват информации;
		\item изменение маршрутов следования информации.
	\end{itemize}
	\item угрозы целостности --- угрозы, при реализации которых информация теряет заранее определённые системой вид и качество. Объектами данной угрозы могут быть все компоненты СЭД:
	\begin{itemize}
		\item документы;
		\item резервные копии документов;
		\item среда хранения электронных документов;
		\item операционные системы и компоненты СЭД, установленные на клиентских рабочих станциях;
		\item каналы связи.
	\end{itemize}
	\item угрозы доступности, характеризующие возможность доступа к хранимой и обрабатываемой в СЭД информации в любой момент времени.
\end{itemize}

\vspace{\baselineskip}
В таблице \ref{table:threats} представлены возможные воздействия на элементы СЭД, приводящие к нарушению их нормального функционирования и, как к одному из сдледствий, потере юридической значимости электронных документов \cite{threats}.
 \begin{center}
 \renewcommand\multirowsetup{\centering}
 \begin{longtable}[h]{| >{\centering}m{5cm} | >{\centering}m{5cm} | >{\centering}m{5cm} |}
	\captionsetup{justification=raggedright}
	\caption{Угрозы безопасности информации в СЭД} \label{table:threats} \tabularnewline
	\hline

 \rowcolor{Gray}   Элемент СЭД & Вид деструктивного воздействия &  Результат воздействия \tabularnewline \hline \endfirsthead   \hline
 \multicolumn{3}{|c|}{\small\slshape (продолжение таблицы \ref{table:threats})}        \tabularnewline \hline
 \rowcolor{Gray}   Элемент СЭД & Вид деструктивного воздействия &  Результат воздействия \tabularnewline \hline
                                              \endhead        \hline
 % \multicolumn{3}{|r|}{\small\slshape продолжение следует}  \tabularnewline \hline
                                              \endfoot        \hline
                                              \endlastfoot

 \multirow{5}{5cm}{Система хранения и обработки документов} & Воздействие на носитель информации & Нарушение целостности и доступности ЭД, истории и метаданных \tabularnewline \cline{2-3}
 		& Внесение искажений в подписанный электронной подписью ЭД & Нарушение целостности ЭД \tabularnewline \cline{2-3} %\pagebreak
 		& Несанкционированное изменение метаданных документа (формат, реквизиты, и т.п.) & Нарушение целостности, возможное нарушение доступности ЭД \tabularnewline \cline{2-3}
 		& Несанкционированное внесение изменений в историю ЭД & Нарушение целостности, возможное нарушение доступности ЭД \tabularnewline \cline{2-3}
 		& Нелегитимное копирование ЭД & Нарушение конфиденциальности ЭД и истории \tabularnewline \hline

 \multirow{2}{5cm}{Система передачи информации} & DoS каналов связи, отказ комплектующих & Нарушение целостности и доступности ЭД. Нарушение доступности CRL и сервера доверенного времени \tabularnewline \cline{2-3}
 		& Внесение искажений в передаваемую информацию & Нарушение целостности ЭД \tabularnewline \hline

 \multirow{2}{5cm}{Система разграничения доступа} & Компрометация аутентификационных данных легитимного пользователя & Нарушение конфиденциальности \tabularnewline \cline{2-3}
 		& Ошибки администрирования (преднамеренные и непреднамеренные): разрешение на доступ нелегитимных пользователей, запрет доступа легитимным пользователям & Нарушение  конфиденциальности. Нарушение доступности \tabularnewline \hline

 {Система проверки подлинности} & Компрометация ключа ЭП либо ключа удостоверяющего центра & {Нарушение целостности ЭД} \tabularnewline \cline{2-2}
 		& Нелигитимное делигирование уполномоченным лицом права подписи ЭД & \tabularnewline \cline{2-3}
 		& Компрометация списка отозванных сертификатов & Нарушение достоверности ЭП \tabularnewline \cline{2-3}
 		& Отказ носителя ключа ЭП & Нарушение доступности ключа ЭП, невозможность штатной работы СЭД \tabularnewline \hline
 
\end{longtable}
\end{center}

% \vspace{\baselineskip}
Защиту от этих угроз в той или иной мере должна реализовывать любая система электронного документооборота. При этом, с одной стороны, при внедрении СЭД увеличиваются риски реализации угроз, но, с другой стороны, при правильном подходе упорядочение документооборота позволяет выстроить более качественную систему защиты.

\vspace{\baselineskip}
Таким образом, любая защищённая СЭД должна предусматривать реализацию как минимум следующих механизмов защиты: 
\begin{itemize}
	\item обеспечение целостности документов;
	\item обеспечение безопасного доступа к компонентам системы;
	\item обеспечение конфиденциальности документов;
	\item обеспечение подлинности документов;
	\item протоколирование действий пользователей.
\end{itemize}

			% Классификация угроз
\subsection{Анализ средств противодействия угрозам информационной безопасности систем электронного документооборота} \label{threats_neutralizers}

Для борьбы с вышеописанными угрозами возможно применение следующих средств:
\begin{enumerate}
	\item \textbf{Воздействие на носитель информации.}

	Для защиты от разрушающих воздействий на хранилища документов необходимо обеспечить их физическую, электромагнитную и вибрационную защиту.

	\item \textbf{Несанкционированное изменение метаданных документа (формат, реквизиты, и т.п.).}

	Для предотвращения этой угрозы необходимо использовать систему полного разграничения доступа с отдельной настройкой по каждому документу / пакету документов, метаданным к ним, и т.п. Контроль за делегированием полномочий, делегирование полномочий с ограничением по времени.

	\item \textbf{Несанкционированное внесение изменений в историю электронного документа.}

	Для защиты от правок записей в истории изменений документа необходимо обеспечить контроль целостности этих записей. В общем случае для этого необходим сервер доверенного времени, с помощью которого для каждой записи в истории будет создаваться метка времени, контролирующая одновременно её целостность и время создания.

	\vspace{\baselineskip}
	Однако, не всегда есть необходимость в установлении точного времени записи: часто достаточно только удостовериться во взаимном положении записей во времени. В таком случае целесообразно не устанавливать сложный в обслуживании сервер доверенного времени, а использовать метод хранения записей в виде цепочки хэш-сумм: в каждой такой записи помимо полезных данных будет содержаться хэш-сумма предыдущей записи. Это позволит упорядочить события во времени, а также усложнит задачу компрометации записей. Подробнее этот метод описан в разделе \ref{research_history}.

	\item \textbf{Нелегитимное копирование электронного документа.}

	Защитой от нелегитимного копирования электронных документов является любое средство защиты информации от НСД --- например, средство шифрования данных в локальном хранилище, средство защиты каналов передачи данных, и т.п. Дополнительным уровнем защиты, применимым также для противодействия внутренним угрозам, является специальное программное обеспечение для чтения документов. Например, вместо предоставления документов по запросу можно предоставлять доступ к удалённому средству просмотра (например, через веб-браузер в защищённой сессии) без возможности локального копирования документов. Для полной защиты необходимо также ограничить круг терминалов удалённого доступа и запретить на них скриншоты (снимки экрана). В случае, если помимо просмотра документа требуется и его редактирование, такое приложение должно предоставлять и данный сервис.

	\item \textbf{Внесение искажений в передаваемую информацию.}

	Так как по каналу передаётся защищённый (зашифрованный) поток данных, в качестве искажений могут рассматриваться только помехи, наводимые на линию связи для нарушения целостности передаваемой информации. Для защиты от такой атаки следует экранировать кабели передачи данных, а также сетевое оборудование.

	\item \textbf{Компрометация аутентификационных данных легитимного пользователя.}

	В общем случае, аутентификационные данные могут быть скомпрометированы путём взлома системы аутентификации, социальной инженерии либо полного перебора паролей. Для защиты от атак по первому вектору необходимо анализировать методы и реализации защиты до принятия решения об их использовании, по второму вектору --- проводить обучающие семинары с персоналом на тему использования среств защиты данных и политик безопасности. Для усложнения перебора паролей рекомендуется заменить средства парольной защиты на системы, использующие асимметричную криптографию. В таком случае сертификат открытого ключа будет являться идентификатором пользователя, а доказательство владения закрытым ключом (proof of knowledge) --- его аутентификатором. Сложность подбора пароля в таком случае возрастёт с $2^{48}$ (при средней длине пароля 8 знаков, без учёта словарного подбора) до $2^{512}$ (по ГОСТ Р 34.10-2012, словари отсутствуют по определению) и станет бессмысленной: для реализации данного вектора атаки злоумышленнику придётся получить доступ к хранилищу ключей, что обычно гораздо более сложная задача.

	\vspace{\baselineskip}
	В ещё более надёжной системе можно использовать асимметричные ключи, записанные на токены \cite{gpghowto, rutoken}. Такие ключи неизвлекаемы, для доступа к ним требуется физический доступ к их носителю --- токен злоумышленнику придётся украсть, а обнаружить пропажу физического ключа проще, чем электронного.

	\item \textbf{Нелегитимное делегирование уполномоченным лицом права подписи электронного документа.}

	Данная атака является следствием нарушения административных регламентов обеспечения информационной безопасности. Бороться с ней неоходимо в двух направлениях: с одной стороны, проводить обучающие семинары по основам и политикам информационной безопасности среди сотрудников. С другой стороны, необходимо по возможности ограничить возможность делегирования полномочий как кругом лиц, которым можно передать права, так и по времени (например, сделать невозможным делегирование завершённой задачи, либо задачи, ожидающей обработки другим редактором).
\end{enumerate}

\vspace{\baselineskip}
Остальные описанные угрозы относятся либо к разряду криптографических проблем общего характера и требуют от разработчиков средств защиты лишь правильного выбора алгоритмов и реализаций (\textit{внесение искажений в подписанный электронной подписью ЭД, компрометация ключа ЭП либо ключа удостоверяющего центра, компрометация списка отозванных сертификатов}), либо является следствием отказа оборудования (\textit{отказ сетевых устройств, отказ носителя ключа ЭП}) или человеческого фактора (\textit{ошибки администрирования}). Методы решения этих проблем носят общий характер и не являются уникальными для систем электронного документооборота 			% Анализ средств противодействия