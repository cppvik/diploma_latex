\subsection{Анализ средств противодействия угрозам информационной безопасности систем электронного документооборота} \label{threats_neutralizers}

Для борьбы с вышеописанными угрозами возможно применение следующих средств:
\begin{enumerate}
	\item \textbf{Воздействие на носитель информации.}

	Для защиты от разрушающих воздействий на хранилища документов необходимо обеспечить их физическую, электромагнитную и вибрационную защиту.

	\item \textbf{Несанкционированное изменение метаданных документа (формат, реквизиты, и т.п.).}

	Для предотвращения этой угрозы необходимо использование системы польного разграничения доступа с отдельной настройкой по каждому документы / пакету документов, метаданным к ним, и т.п. Контроль за делигированием полномочий, делегирование полномочий с ограничением по времени.

	\item \textbf{Несанкционированное внесение изменений в историю ЭД.}

	Для защиты от правок записей в истории изменений документа необходимо обеспечить контроль целостности этих записей. В общем случае для этого необходим сервер доверенного времени, с помощью которого для каждой записи в истории будет создаваться метка времени, контролирующая одновременно её целостность и время создания.

	Однако, не всегда есть необходимость в установлении точного времени записи: часто достаточно только удостовериться во взаимном положении записей во времени. В таком случае целесообразно не устанавливать сложный в обслуживании сервер доверенного времени, а использовать метод хранения записей в виде цепочки хэш-сумм: в каждой такой записи помимо полезных данных будет содержаться хэш-сумма предыдущей записи. Это позволит упорядочить события во времени, а также усложнит задачу компрометации записей. Подробнее этот метод описан в разделе \ref{research_history}.

	\item \textbf{Нелигитимное копирование электронного документа.}

	Защитой от нелигитимного копирования электронных документов является любое средство защиты информации от НСД --- например, средтсво шифрования данных на локальном хранилище, средство защиты каналов передачи данных, и т.п. Дополнительным уровнем защиты, применимым также для противодействия внутренним угрозам, является специальное ПО для чтения документов. Например, вместо предоставления документов по запросу можно предоставлять доступ к удалённому средству просмотра (например, через веб-браузер в защищённой сессии) без возможности локального копирования документов. Для полной защиты необходимо также ограничить круг терминалов удалённого доступа и запретить на них скриншоты (снимки экрана). В случае, если помимо просмотра документа требуется и его редактирование, такое приложение должно предоставлять и данный сервис.

	\item \textbf{Внесение искажений в передаваемую информацию.}

	Так как по каналу передаётся защищённый (зашифрованный) поток данных, в качестве искажений могут рассматриваться только помехи, наводимые на линию связи для нарушения целостности передаваемой информации. Для защиты от такой атаки следует экранировать кабели передачи данных, а также сетевое оборудование.

	\item \textbf{Компрометация аутентификационных данных легитимного пользователя.}

	В общем случае, аутентификационные данные могут быть скомпрометированы путём взлома системы аутентификации, социальной инженерии либо полного перебора паролей. Для защиты от атак по первому вектору необходимо анализировать методы и реализации защиты до принятия решения об их использовании, по второму вектору --- проводить обучающие семинары с персоналом на тему использования среств защиты данных и политик безопасности. Для усложнения перебора паролей рекомендуется заменить средства парольной защиты на системы, использующие асимметричную криптографию. В таком случае сертификат открытого ключа будет являться идентификатором пользователя, а доказательство владения закрытым ключом (proof of knowledge) --- его аутентификатором. Сложность подбора пароля в таком случае возрастёт с $2^{48}$ (при средней длине пароля 8 знаков, без учёта словарного подбора) до $2^{512}$ (по ГОСТ Р 34.10-2012, словари отсутствуют по определению) и станет бессмысленной: для реализации данного вектора атаки злоумышленнику придётся получить доступ к хранилищу ключей, что обычно гораздо более сложная задача.

	В ещё более надёжной системе можно использовать асимметричные ключи, записанные на токены. Такие ключи неизвлекаемы, и для доступа к ним требуется физический доступ к их носителю --- токен злоумышленнику придётся украсть, а обнаружить пропажу физического ключа проще, чем электронного.

	\item \textbf{Нелигитимное делигирование уполномоченным лицом права подписи ЭД.}

	Данная атака является следствием нарушения административных регламентов обеспечения информационной безопасности. Бороться с ней неоходимо в двух направлениях: с одной стороны, проводить обучающие семинары по основам и политикам информационной безопасности среди сотрудников. С другой стороны, необходимо по возможности ограничить возможность делигирования полномочий как кругом лиц, которым можно передать права, так и по времени (например, сделать невозможным делигирование завершённой задачи, либо задачи, ожидающей обработки другим редактором).
\end{enumerate}

\vspace{\baselineskip}
Остальные описанные угрозы относятся либо к разряду криптографических проблем общего характера и требуют от разработчиков средств защиты лишь правильного выбора алгоритмов и реализаций (\textit{внесение искажений в подписанный электронной подписью ЭД, компрометация ключа ЭП либо ключа удостоверяющего центра, компрометация списка отозванных сертификатов}), либо является следствием отказа оборудования (\textit{отказ сетевых устройств, отказ носителя ключа ЭП}) или человеческого фактора (\textit{ошибки администрирования}). Методы решения этих проблем носят общий характер и не уникальны для систем электронного документооборота 