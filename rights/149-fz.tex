\section{Федеральный закон <<Об информации, информационных технологиях и о защите информации>>} \label{rights_149}

Федеральный закон №149\ndashФЗ <<Об информации, информационных технологиях и о защите информации>>, принятый Государственной Думой 8 июля 2006 года, одобренный Советом Федерации 14 июля 2006 года, вступивший в силу 27 июля 2006 года после публикации в Российской газете (№4131), регулирует отношения, возникающие при:
\begin{itemize}
	\item осуществлении права на поиск, получение, передачу, производство и распространение информации;
	\item применении информационных технологий;
	\item обеспечении защиты информации.
\end{itemize}

Положения Федерального закона не распространяются на отношения, возникающие при правовой охране результатов интеллектуальной деятельности и приравненных к ним средств индивидуализации.
\vspace{\baselineskip}
Согласно ФЗ №149, правовое регулирование отношений, возникающих в сфере информации, информационных технологий и защиты информации, основывается на следующих принципах:
\begin{itemize}
	\item свобода поиска, получения, передачи, производства и распространения информации любым законным способом;
	\item установление ограничений доступа к информации только федеральными законами;
	\item открытость информации о деятельности государственных органов и органов местного самоуправления и свободный доступ к такой информации, кроме случаев, установленных федеральными законами;
	\item равноправие языков народов Российской Федерации при создании информационных систем и их эксплуатации;
	\item обеспечение безопасности Российской Федерации при создании информационных систем, их эксплуатации и защите содержащейся в них информации;
	\item достоверность информации и своевременность ее предоставления;
	\item неприкосновенность частной жизни, недопустимость сбора, хранения, использования и распространения информации о частной жизни лица без его согласия;
	\item недопустимость установления нормативными правовыми актами каких\ndashлибо преимуществ применения одних информационных технологий перед другими, если только обязательность применения определенных информационных технологий для создания и эксплуатации государственных информационных систем не установлена федеральными законами.
\end{itemize}

Федеральный закон гласит, что информация может являться объектом публичных, гражданских и иных правовых отношений. Информация может свободно использоваться любым лицом и передаваться одним лицом другому лицу, если федеральными законами не установлены ограничения доступа к информации либо иные требования к порядку ее предоставления или распространения. Информация в зависимости от категории доступа к ней подразделяется на общедоступную информацию, а также на информацию, доступ к которой ограничен федеральными законами (информация ограниченного доступа).

\vspace{\baselineskip}
Ограничение доступа к информации устанавливается федеральными законами в целях защиты основ конституционного строя, нравственности, здоровья, прав и законных интересов других лиц, обеспечения обороны страны и безопасности государства. Обязательным является соблюдение конфиденциальности информации, доступ к которой ограничен федеральными законами.

\vspace{\baselineskip}
Защита информации, составляющей государственную тайну, осуществляется в соответствии с законодательством Российской Федерации о государственной тайне. Федеральными законами устанавливаются условия отнесения информации к сведениям, составляющим коммерческую тайну, служебную тайну и иную тайну, обязательность соблюдения конфиденциальности такой информации, а также ответственность за ее разглашение.

\vspace{\baselineskip}
Статья 16 Федерального закона <<Об информации, информационных технологиях и о защите информации>> посвящена защите информации. Согласно ей защита информации представляет собой принятие правовых, организационных и технических мер, направленных на:
\begin{itemize}
	\item обеспечение защиты информации от неправомерного доступа, уничтожения, модифицирования, блокирования, копирования, предоставления, распространения, а также от иных неправомерных действий в отношении такой информации;
	\item соблюдение конфиденциальности информации ограниченного доступа,
	\item реализацию права на доступ к информации.
\end{itemize}

Государственное регулирование отношений в сфере защиты информации осуществляется путем установления требований о защите информации, а также ответственности за нарушение законодательства Российской Федерации об информации, информационных технологиях и о защите информации. 

\vspace{\baselineskip}
Нарушение требований настоящего Федерального закона влечет за собой дисциплинарную, гражданско\ndashправовую, административную или уголовную ответственность в соответствии с законодательством Российской Федерации.