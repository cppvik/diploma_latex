\section{Федеральный закон <<О персональных данных>>} \label{rights_152}

Федеральный закон №152-ФЗ <<О персональных данных>> принят Государственной Думой 08 июля 2006 года, одобрен Советом Федерации 14 июля 2006 года и вступил в силу 26 января 2007 года после публикации в Российской газете (№4131). Данный закон направлен на реализацию конституционных положений, закрепляющих право каждого на неприкосновенность частной жизни и свободу информации, а также международных обязательств Российской Федерации по ратификации Конвенции Совета Европы о защите физических лиц при автоматизированной обработке персональных данных.

\vspace{\baselineskip}
В законе устанавливаются общие унифицированные требования к обработке персональных данных во всех сферах, где используются эти данные, определяются права субъектов персональных данных и обязанности операторов, осуществляющих обработку данных, принципы трансграничной передачи персональных данных, а также меры государственного контроля за деятельностью государственных и муниципальных органов, юридических и физических лиц, связанной с обработкой персональных данных.

\vspace{\baselineskip}
Федеральным законом допускаются различные способы учета персональных данных в государственных и муниципальных информационных системах, в том числе различные способы обозначения принадлежности персональных данных конкретному лицу.

В государственных и муниципальных информационных системах предусматривается возможность создания государственного регистра населения, правовой статус и порядок работы с которым устанавливаются федеральным законом.

Обеспечение контроля и надзора за соответствием обработки персональных данных возлагается на уполномоченный орган по защите прав субъектов персональных данных, которым является федеральный орган исполнительной власти, осуществляющий функции по контролю и надзору в сфере информационных технологий и связи.

\vspace{\baselineskip}
Россвязькомнадзор осуществляет контроль и надзор за соответствием обработки ПДн требованиям законодательства.
ФСТЭК России устанавливает методы и способы защиты информации в информационных системах в пределах своих полномочий.
ФСБ России устанавливает методы и способы защиты информации в информационных системах в пределах своих полномочий (регулирует сферу использования криптографических средств защиты информации).