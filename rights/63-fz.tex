\section{Федеральный закон <<Об электронной подписи>>} \label{rights_63}

Федеральный закон №63-ФЗ <<Об электронной подписи>>, принятый Государственной Думой 25 марта 2011 года, одобренный Советом Федерации 30 марта 2011 года, вступивший в силу 8 апреля 2011 года после публикации в Российской газете (№5451) \cite{rg-63fz}, регулирует отношения  в области использования электронных подписей при совершении гражданско-правовых сделок, оказании государственных и муниципальных услуг, исполнении государственных и муниципальных функций, при совершении иных юридически значимых действий.

\vspace{\baselineskip}
В ФЗ №63 используются следующие понятия:
\begin{itemize}
	\item электронная подпись - информация в электронной форме, которая присоединена к другой информации в электронной форме (подписываемой информации) или иным образом связана с такой информацией и которая используется для определения лица, подписывающего информацию;
	\item сертификат ключа проверки электронной подписи - электронный документ или документ на бумажном носителе, выданные удостоверяющим центром либо доверенным лицом удостоверяющего центра и подтверждающие принадлежность ключа проверки электронной подписи владельцу сертификата ключа проверки электронной подписи;
	\item квалифицированный сертификат ключа проверки электронной подписи (далее - квалифицированный сертификат) - сертификат ключа проверки электронной подписи, выданный аккредитованным удостоверяющим центром или доверенным лицом аккредитованного удостоверяющего центра либо федеральным органом исполнительной власти, уполномоченным в сфере использования электронной подписи (далее - уполномоченный федеральный орган);
	\item владелец сертификата ключа проверки электронной подписи - лицо, которому в установленном настоящим Федеральным законом порядке выдан сертификат ключа проверки электронной подписи;
	\item ключ электронной подписи - уникальная последовательность символов, предназначенная для создания электронной подписи;
	\item ключ проверки электронной подписи - уникальная последовательность символов, однозначно связанная с ключом электронной подписи и предназначенная для проверки подлинности электронной подписи (далее - проверка электронной подписи);
	\item удостоверяющий центр - юридическое лицо или индивидуальный предприниматель, осуществляющие функции по созданию и выдаче сертификатов ключей проверки электронных подписей, а также иные функции, предусмотренные настоящим Федеральным законом;
	\item аккредитация удостоверяющего центра - признание уполномоченным федеральным органом соответствия удостоверяющего центра требованиям настоящего Федерального закона;
	\item средства электронной подписи - шифровальные (криптографические) средства, используемые для реализации хотя бы одной из следующих функций - создание электронной подписи, проверка электронной подписи, создание ключа электронной подписи и ключа проверки электронной подписи;
	\item средства удостоверяющего центра - программные и (или) аппаратные средства, используемые для реализации функций удостоверяющего центра;
	\item участники электронного взаимодействия - осуществляющие обмен информацией в электронной форме государственные органы, органы местного самоуправления, организации, а также граждане;
	\item корпоративная информационная система - информационная система, участники электронного взаимодействия в которой составляют определенный круг лиц;
	\item информационная система общего пользования - информационная система, участники электронного взаимодействия в которой составляют неопределенный круг лиц и в использовании которой этим лицам не может быть отказано.
\end{itemize}

\vspace{\baselineskip}
Статья 6 определяет условия признания электронных документов, подписанных электронной подписью, равнозначными документам на бумажном носителе, подписанным собственноручной подписью:
\begin{itemize}
	\item Информация в электронной форме, подписанная квалифицированной электронной подписью, признается электронным документом, равнозначным документу на бумажном носителе, подписанному собственноручной подписью, кроме случая, если федеральными законами или принимаемыми в соответствии с ними нормативными правовыми актами установлено требование о необходимости составления документа исключительно на бумажном носителе.
	\item Информация в электронной форме, подписанная простой электронной подписью или неквалифицированной электронной подписью, признается электронным документом, равнозначным документу на бумажном носителе, подписанному собственноручной подписью, в случаях, установленных федеральными законами, принимаемыми в соответствии с ними нормативными правовыми актами или соглашением между участниками электронного взаимодействия.
	\item Если в соответствии с федеральными законами, принимаемыми в соответствии с ними нормативными правовыми актами или обычаем делового оборота документ должен быть заверен печатью, электронный документ, подписанный усиленной электронной подписью и признаваемый равнозначным документу на бумажном носителе, подписанному собственноручной подписью, признается равнозначным документу на бумажном носителе, подписанному собственноручной подписью и заверенному печатью.
	\item Одной электронной подписью могут быть подписаны несколько связанных между собой электронных документов (пакет электронных документов). При подписании электронной подписью пакета электронных документов каждый из электронных документов, входящих в этот пакет, считается подписанным электронной подписью того вида, которой подписан пакет электронных документов.
\end{itemize}

В статье 9 определяются условия, при выполнении которых электронный документ считается подписанным простой электронной подписью:
\begin{itemize}
	\item простая электронная подпись содержится в самом электронном документе;
	\item ключ простой электронной подписи применяется в соответствии с правилами, установленными оператором информационной системы, с использованием которой осуществляются создание и (или) отправка электронного документа, и в созданном и (или) отправленном электронном документе содержится информация, указывающая на лицо, от имени которого был создан и (или) отправлен электронный документ.
\end{itemize}

Таким образом, ФЗ №63 определяет условия, при которых бумажный документооборот может быть заменён электронным.