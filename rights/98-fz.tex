\section{Федеральный закон <<О коммерческой тайне>>} \label{rights_98}

Федеральный закон №98-ФЗ <<О коммерческой тайне>> принят Государственной Думой 09 июля 2004 года, одобрен Советом Федерации 15 июля 2004 года и вступил в силу 16 августа 2004 года после публикации в Российской газете (№3543).
Цель данного закона -- ограничить нелегальный отток служебной информации коммерческого характера от одной компании к другой.

\vspace{\baselineskip}
Коммерческая тайна определяется как конфиденциальность информации, позволяющая ее обладателю при существующих или возможных обстоятельствах увеличить доходы, избежать неоправданных расходов, сохранить положение на рынке товаров, работ, услуг или получить иную коммерческую выгоду.
Закон предусматривает, что обладатель коммерческой тайны имеет право устанавливать, изменять и отменять в письменной форме режим коммерческой тайны.

Закон также устанавливает, что срок неразглашения работником информации, составляющей коммерческую тайну определяется соглашением между сотрудником и компанией, если такое соглашение не подписывалось, срок составляет три года с момента прекращения действия трудового договора. Помимо этого в документе вводится понятие ответственности за разглашение коммерческой тайны и за непредставление органам государственной власти и местного самоуправления информации, составляющей коммерческую тайну, по специальному требованию.

\vspace{\baselineskip}
Закон вводит то, что пострадавшая от разглашения коммерческой тайны компания сможет в Арбитражном суде потребовать возмещения всего доказанного ущерба у компании, которая воспользовался утечкой, или у представителей власти, виновных в утечке секретов.

\vspace{\baselineskip}
В тексте документа оговаривается, что коммерческую тайну не могут составлять сведения о загрязнении окружающей среды, состоянии противопожарной безопасности, санитарно-эпидемиологической и радиационной обстановке, качестве пищевых продуктов и других факторах, связанных с обеспечением безопасности населения. Кроме того, не может засекречиваться информация, имеющаяся в учредительных документах юридического лица, о численности и составе работников организаций, о системе оплаты труда и задолженности работодателей по выплате зарплаты. Помимо этого в законе установлен круг лиц, которым предприятия обязаны предоставлять информацию, относящуюся к разряду коммерческой тайны. Это касается запросов со стороны судов и органов прокуратуры. Круг лиц, обязанных осуществлять меры по охране конфиденциальности информации, дополнен индивидуальными предпринимателями.

