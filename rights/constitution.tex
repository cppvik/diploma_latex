\section{Конституция Российской Федерации} \label{rights_constitution}

Конституция Российской Федерации, принятая на всенародном голосовании 12 декабря 1993 года, имеет высшую юридическую силу и прямое действие и применяется на всей территории Российской Федерации. Она является основным источником права в Российской Федерации, в том числе в области обеспечения информационной безопасности Российской Федерации.

\vspace{\baselineskip}
Согласно статье 23 Конституции Российской Федерации, каждый имеет право на тайну переписки, телефонных переговоров, почтовых, телеграфных и иных сообщений. Данная статья подтверждает необходимость защиты информации, передаваемой по открытым каналам связи.

\vspace{\baselineskip}
Статья 29 гласит, что каждый имеет право свободно искать, получать, передавать, производить и распространять информацию любым законным способом. Перечень сведений, составляющих государственную тайну, определяется федеральным законом. Для обеспечения этого права необходимо позаботиться о доступности информации.
