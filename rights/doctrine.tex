\section{Доктрина информационной безопасности Российской Федерации} \label{rights_doctrine}

Доктрина информационной безопасности Российской Федерации, принятая 23 июня 2000 года Советом безопасности РФ, представляет собой совокупность официальных взглядов на цели, задачи, принципы и основные направления обеспечения информационной безопасности Российской Федерации.

\vspace{\baselineskip}
Доктрина информационной безопасности служит основой для:
\begin{itemize}
	\item формирования государственной политики в области обеспечения информационной безопасности Российской Федерации; 
	\item подготовки предложений по совершенствованию правового, методического, научно\ndash технического и организационного обеспечения информационной безопасности Российской Федерации; 
	\item разработки целевых программ обеспечения информационной безопасности Российской Федерации. 
\end{itemize}

Доктрина информационной безопасности Российской Федерации развивает Концепцию национальной безопасности Российской Федерации применительно к информационной сфере.

\vspace{\baselineskip}
В соответствии с доктриной информационной безопасности Российской Федерации, основными объектами обеспечения информационной безопасности в общегосударственных информационных и телекоммуникационных системах являются:
\begin{itemize}
	\item информационные ресурсы, содержащие конфиденциальную информацию и сведения, отнесённые к государственной тайне;
	\item средства и системы информатизации (средства вычислительной техники, информационно\ndash вычислительные комплексы, сети и системы), программные средства (операционные системы, системы управления базами данных, другое общесистемное и прикладное программное обеспечение), автоматизированные системы управления, системы связи и передачи данных, осуществляющие прием, обработку, хранение и передачу информации ограниченного доступа, их информативные физические поля;
	\item технические средства и системы, обрабатывающие открытую информацию, но размещённые в помещениях, в которых обрабатывается информация ограниченного доступа, а также сами помещения, предназначенные для обработки такой информации;
	\item помещения, предназначенные для ведения закрытых переговоров, а также переговоров, в ходе которых оглашаются сведения ограниченного доступа.
\end{itemize}

Основными угрозами информационной безопасности Российской Федерации в общегосударственных информационных и телекоммуникационных системах являются:
\begin{itemize}
	\item деятельность специальных служб иностранных государств, преступных сообществ, организаций и групп, противозаконная деятельность отдельных лиц, направленная на получение несанкционированного доступа к информации и осуществление контроля за функционированием информационных и телекоммуникационных систем;
	\item вынужденное в силу объективного отставания отечественной промышленности использование при создании и развитии информационных и телекоммуникационных систем импортных программно\ndashаппаратных средств;
	\item нарушение установленного регламента сбора, обработки и передачи информации, преднамеренные действия и ошибки персонала информационных и телекоммуникационных систем, отказ технических средств и сбои программного обеспечения в информационных и телекоммуникационных системах;
	\item использование не сертифицированных в соответствии с требованиями безопасности средств и систем информатизации и связи, а также средств защиты информации и контроля их эффективности;
	\item привлечение к работам по созданию, развитию и защите информационных и телекоммуникационных систем организаций и фирм, не имеющих государственных лицензий на осуществление этих видов деятельности.
\end{itemize}

Основными направлениями обеспечения информационной безопасности Российской Федерации в общегосударственных информационных и телекоммуникационных системах являются:
\begin{itemize}
	\item предотвращение перехвата информации из помещений и с объектов, а также информации, передаваемой по каналам связи с помощью технических средств;
	\item исключение несанкционированного доступа к обрабатываемой или хранящейся в технических средствах информации;
	\item предотвращение утечки информации по техническим каналам, возникающей при эксплуатации технических средств ее обработки, хранения и передачи;
	\item предотвращение специальных программно\ndashтехнических воздействий, вызывающих разрушение, уничтожение, искажение информации или сбои в работе средств информатизации;
	\item обеспечение информационной безопасности при подключении общегосударственных информационных и телекоммуникационных систем к внешним информационным сетям, включая международные;
	\item обеспечение безопасности конфиденциальной информации при взаимодействии информационных и телекоммуникационных систем различных классов защищенности;
	\item выявление внедренных на объекты и в технические средства электронных устройств перехвата информации.
\end{itemize}

Основными организационно\ndashтехническими мероприятиями по защите информации в общегосударственных информационных и телекоммуникационных системах являются:
\begin{itemize}
	\item лицензирование деятельности организаций в области защиты информации;
	\item аттестация объектов информатизации по выполнению требований обеспечения защиты информации при проведении работ, связанных с использованием сведений, составляющих государственную тайну;
	\item сертификация средств защиты информации и контроля эффективности их использования, а также защищенности информации от утечки по техническим каналам систем и средств информатизации и связи;
	\item введение территориальных, частотных, энергетических, пространственных и временных ограничений в режимах использования технических средств, подлежащих защите;
	\item создание и применение информационных и автоматизированных систем управления в защищенном исполнении.
\end{itemize}