\section{Приказ ФСБ РФ <<Об утверждении Положения о разработке, производстве, реализации и эксплуатации шифровальных (криптографических) средств защиты информации (Положение ПКЗ-2005)>>} \label{rights_fsb_66}

Приказ ФСБ РФ от 09 февраля 2005 года №66 <<Об утверждении Положения о разработке, производстве, реализации и эксплуатации шифровальных (криптографических) средств защиты информации (Положение ПКЗ-2005)>>, вступивший в силу через 10 дней после публикации в Российской газете (№55 от 19 марта 2005 года), создан с целью определения
порядка разработки, производства, шифровальных (криптографических) ограниченным доступом, не реализации средств содержащей государственную тайну.

\vspace{\baselineskip}
Положением необходимо руководствоваться при разработке, производстве, реализации и эксплуатации средств криптографической защиты информации конфиденциального характера в следующих случаях:
\begin{itemize}
	\item если информация конфиденциального характера подлежит защите в соответствии с законодательством Российской Федерации;
	\item при организации криптографической защиты информации конфиденциального характера в федеральных органах исполнительной власти, органах исполнительной власти субъектов Российской Федерации (далее - государственные органы);
	\item при организации криптографической защиты информации конфиденциального характера в организациях независимо от их организационно-правовой формы и формы собственности при выполнении ими заказов на поставку товаров, выполнение работ или оказание услуг для государственных нужд (далее --- организации, выполняющие государственные заказы);
	\item если обязательность защиты информации конфиденциального характера возлагается законодательством Российской Федерации на лиц, имеющих доступ к этой информации или наделенных полномочиями по распоряжению сведениями, содержащимися в данной информации;
	\item при обрабатывании информации конфиденциального характера, обладателем которой являются государственные органы или организации, выполняющие государственные заказы, в случае принятия ими мер по охране ее конфиденциальности путем использования средств криптографической защиты;
	\item при обрабатывании информации конфиденциального характера в государственных органах и в организациях, выполняющих государственные заказы, обладатель которой принимает меры к охране ее конфиденциальности путем установления необходимости криптографической защиты данной информации.
\end{itemize}

Положение регулирует порядок разработки, производства, реализации и эксплуатации системы криптографической защиты информации.
