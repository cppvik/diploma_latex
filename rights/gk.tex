\section{Гражданский кодекс Российской Федерации. Часть четвёртая} \label{rights_gk}

Гражданский кодекс Российской Федерации, вступивший в силу 18 декабря 2006 года, в части четвертой регулирует вопросы охраны результатов интеллектуальной деятельности и  средств индивидуализации. Согласно ГК РФ, статья 1225, к результатам интеллектуальной деятельности, которым предоставляется правовая охрана, в числе прочих относятся: произведения науки и программы для электронных вычислительных машин (ЭВМ).

\vspace{\baselineskip}
Статья 1228 гласит, что автором результата интеллектуальной деятельности признается гражданин, творческим трудом которого создан такой результат. Не признаются авторами результата интеллектуальной деятельности граждане, не внесшие личного творческого вклада в создание такого результата, в том числе оказавшие его автору только техническое, консультационное, организационное или материальное содействие или помощь либо только способствовавшие оформлению прав на такой результат или его использованию, а также граждане, осуществлявшие контроль за выполнением соответствующих работ.

\vspace{\baselineskip}
Согласно статье 1259, объектами авторских прав являются произведения науки, литературы и искусства независимо от достоинств и назначения произведения, а также от способа его выражения. К объектам авторских прав также относятся программы для ЭВМ, которые охраняются как литературные произведения. Для возникновения, осуществления и защиты авторских прав не требуется регистрация произведения или соблюдение каких-либо иных формальностей. В отношении программ для ЭВМ и баз данных возможна регистрация, осуществляемая по желанию правообладателя в соответствии с правилами статьи 1262 ГК РФ. Авторские права не распространяются на идеи, концепции, принципы, методы, процессы, системы, способы, решения технических, организационных или иных задач, открытия, факты, языки программирования.

\vspace{\baselineskip}
Таким образом, для объектов авторского права важно иметь средства подтверждения авторства и времени создания.