\section{ГОСТ Р ИСО 15489-1 --- 2007} \label{rights_gost_15489-1}

ГОСТ Р ИСО 15489-1 --- 2007 <<Система стандартов по информации, библиотечному и издательскому делу. Управление документами. Общие требования>> регулирует процессы управления документами государственных или коммерческих организаций, предназначаемыми для внутреннего или внешнего пользования.

В статье 4 описываются преимущества управления документами. Так, документы содержат информацию, являющуюся ценным ресурсом и важным элементом деловой деятельности. Системный подход к управлению документами позволяет организациям и обществу защищать и сохранять документы в качестве доказательства действий. Система управления документами позволяет создать информационный ресурс о деловой деятельности, который может поддерживать последующую деятельность и отдельные решения, а также обеспечивать подотчетность всем заинтересованным сторонам. Данная статья аргументирует необходимость хранения полной и достоверной истории изменения документа.

Статья 7.2 <<Характеристики документа>> описывает признаки надёжного документа:
\begin{itemize}
	\item Документ является аутентичным, если он:
	\begin{enumerate}
	\item является тем, чем должны быть;
	\item был создан или отправлен лицом, уполномоченным на это;
	\item был создан или отправлен в то время, которое обозначено в документе.
	\end{enumerate}
	\item Достоверным является документ, содержание которого можно считать полным и точным представлением подтверждаемых операций, деятельности или фактов и которому можно доверять в последующих операциях или в последующей деятельности. Документы должны создаваться во время или сразу же после операции или случая, к которым они относятся, лицами, достоверно знающими факты, или средствами, обычно используемыми в деловой деятельности при проведении данной операции.
	\item Целостность документа определяется его полнотой и неизменностью.
	\item Пригодным для использования является документ, который можно локализовать, найти, воспроизвести и интерпретировать.
\end{itemize}