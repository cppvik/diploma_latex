\section{Введение} \label{rights_introduction}

Информационная безопасность Российской Федерации является одной из составляющих национальной безопасности и оказывает влияние на защищённость национальных интересов Российской Федерации в различных сферах жизнедеятельности общества и государства. Угрозы информационной безопасности Российской Федерации и методы её обеспечения являются общими для этих сфер.

\vspace{\baselineskip}
В каждой из них имеются свои особенности обеспечения информационной безопасности, связанные со спецификой объектов обеспечения безопасности, степенью их уязвимости в отношении угроз информационной безопасности Российской Федерации. В каждой сфере жизнедеятельности общества и государства наряду с общими методами обеспечения информационной безопасности Российской Федерации могут использоваться частные методы и формы, обусловленные спецификой факторов, влияющих на состояние информационной безопасности Российской Федерации.

\vspace{\baselineskip}
Законодательные меры по защите информации предусматривают создание в стране законодательной базы, предусматривающей разработку новых или корректировку существующих законов, положений, постановлений и инструкций, а также создание действенной системы контроля над исполнением требований в указанных документах.

\vspace{\baselineskip}
Правовое обеспечение информационной безопасности Российской Федерации представляет собой систему правового регулирования общественных отношений в области противодействия угрозам национальных интересов Российской Федерации в информационной сфере. 	Оно включает в себя согласованную систему нормативных актов, регулирующих рассматриваемые отношения, а также согласованную деятельность органов государственной власти по их развитию и совершенствованию.