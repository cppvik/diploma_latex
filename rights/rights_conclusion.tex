\section{Выводы} \label{rights_conclusion}

Нормативно-правовые акты различного уровня устанавливают условия, при которых бумажный документооборот может быть заменён электронным. Одно из важнейших условий -- обеспечение информационной безопасности обрабатываемых документов. Необходимо как обеспечить юридическую значимость документов (это можно сделать с помощью СКЗИ, предусматривающих работу с электронной подписью), так и выполнить требования по защите конфиденциальной, коммерческой, и иной информации, а также персональных данных согласно соответствующим нормативно-правовым актам.

Кроме того, ГОСТ Р ИСО 15489-1 устанавливает, что в процессе документооборота должна полно и достоверно храниться история изменения документа. Этот вопрос -- один из тех, которые не решаются современными системами электронного документооборота, что является обоснованием для новой разработки.

\vspace{\baselineskip}
При разработке ПО для системы электронного документооборота необходимо опираться на вышеперечисленные нормативно-правовые акты.