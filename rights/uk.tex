\section{Уголовный кодекс Российской Федерации} \label{rights_uk}

Уголовный кодекс Российской Федерации, вступивший в силу 13 июня 1996 года, ставит своей целью охрану прав и свобод человека и гражданина, собственности, общественного порядка и общественной безопасности, окружающей среды, конституционного строя Российской Федерации от преступных посягательств, обеспечение мира и безопасности человечества, а также предупреждение преступлений. В частности в нем рассматриваются вопросы обеспечения информационной безопасности. Статья 272 Уголовного Кодекса РФ предусматривает ответственность за неправомерный доступ к компьютерной информации (информации на машинном носителе, в ЭВМ или сети ЭВМ), если это повлекло уничтожение, блокирование модификацию либо копирование информации, нарушение работы вычислительной системы.

\vspace{\baselineskip}
Данная статья защищает право владельца на неприкосновенность информации в системе. Владельцем информационной системы может быть любое лицо, правомерно пользующееся услугами по обработке информации как собственник вычислительной системы или как лицо, приобретшее право использования компьютера.

\vspace{\baselineskip}
Преступное деяние, ответственность за которое предусмотрено ст. 272, состоит в неправомерном доступе к охраняемой законом компьютерной информации, который может выражаться в проникновении в компьютерную систему путем использования специальных технических или программных средств, позволяющих преодолеть установленные защиты, незаконного применения действующих паролей или маскировка под видом законного пользователя для проникновения в компьютер, хищения носителей информации, при условии, что были приняты меры их охраны, если это деяние повлекло уничтожение или блокирование информации.

\vspace{\baselineskip}
По уголовному законодательству субъектами компьютерных преступлений могу быть лица, достигшие 16-летнего возраста, однако часть вторая статьи 272 предусматривает наличие дополнительного признака к субъекта, совершившего данное деяние --- служебное положение, а равно доступ к ЭВМ, системе ЭВМ или сети ЭВМ, способствовавших его совершению.

\vspace{\baselineskip}
Статья 272 УК не регулирует ситуацию, когда неправомерный доступ осуществляется в результате неосторожных действий.