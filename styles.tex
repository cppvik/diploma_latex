%%% Макет страницы %%%
\geometry{a4paper,top=2cm,bottom=2cm,left=2.5cm,right=1cm}
% \geometry{a4paper,top=2cm,bottom=2cm,left=3cm,right=1cm}

%%% Кодировки и шрифты %%%
\renewcommand{\rmdefault}{ftm}			% Включаем Times New Roman
% \renewcommand{\baselinestretch}{1.5}	% Полуторный интервал для всего текста
\onehalfspacing							% Полуторный интервал для всего текста

%%% Выравнивание и переносы %%%
\sloppy					% Избавляемся от переполнений
\clubpenalty=10000		% Запрещаем разрыв страницы после первой строки абзаца
\widowpenalty=10000		% Запрещаем разрыв страницы после последней строки абзаца
% \setlength{\parskip}{\baselineskip}

%%% Библиография %%%
\makeatletter
\bibliographystyle{utf8gost705u}	% Оформляем библиографию в соответствии с ГОСТ 7.0.5
\renewcommand{\@biblabel}[1]{#1}	% Заменяем библиографию с квадратных скобок на точку:
\makeatother

%%% Изображения %%%
\graphicspath{{./images/}} % Пути к изображениям

%%% Цвета гиперссылок %%%
\definecolor{linkcolor}{rgb}{0.0,0,0}
\definecolor{citecolor}{rgb}{0,0.0,0}
\definecolor{urlcolor}{rgb}{0,0,0}
\hypersetup{
    colorlinks, linkcolor={linkcolor},
    citecolor={citecolor}, urlcolor={urlcolor}
}

%%% Оглавление %%%
\renewcommand{\cftchapdotsep}{\cftdotsep}

%%% Переносы %%%
\begingroup
    \lccode`\==`\-
    % Пример: задать правила переноса для `коммерческо-посредническая':
    \hyphenation{бе-зо-пас-ность}
    \hyphenation{бе-зо-пас-ности}
    \hyphenation{бе-зо-пас-ностью}
\endgroup
\newcommand{\ndash}{\nobreakdash--} % неразрывный перенос

%%% Подписи рисунков и таблиц %%%
\DeclareCaptionLabelSeparator{defffis}{ --- }
\captionsetup{justification=centering,labelsep=defffis}

%%% Тире в перечислениях %%%%
\renewcommand{\labelitemi}{\normalfont\bfseries{--}}

%%% Переносы математики %%%
\newcommand{\hm}[1]{#1\nobreak\discretionary{}{\hbox{\ensuremath{#1}}}{}}
\begingroup
\catcode`\+\active\gdef+{\mathchar8235\nobreak\discretionary{}%
{\usefont{OT1}{cmr}{m}{n}\char43}{}}
\catcode`\-\active\gdef-{\mathchar8704\nobreak\discretionary{}%
{\usefont{OMS}{cmsy}{m}{n}\char0}{}}
\catcode`\=\active\gdef={\mathchar12349\nobreak\discretionary{}%
{\usefont{OT1}{cmr}{m}{n}\char61}{}}
\catcode`\<\active\gdef<{\mathchar"313C\nobreak\discretionary{}%
{\usefont{OML}{cmm}{m}{n}\char60}{}}
\catcode`\>\active\gdef>{\mathchar"313E\nobreak\discretionary{}%
{\usefont{OML}{cmm}{m}{n}\char62}{}}
\endgroup
\def\times{\mathchar8706\nobreak\discretionary{}{\usefont{OMS}{cmsy}{m}{n}\char
2}{}}
\def\subset{\mathchar"321A\nobreak\discretionary{}%
{\usefont{OMS}{cmsy}{m}{n}\char26}{}}
%\supset,\subseteq,\notin
\def\neq{\not=\nobreak\discretionary{}%
{\usefont{OMS}{cmsy}{m}{n}\char54\usefont{OT1}{cmr}{m}{n}\char61}{}}
\def\sim{\mathchar"3218\nobreak\discretionary{}%
{\usefont{OMS}{cmsy}{m}{n}\char24}{}}
\def\in{\mathchar"3232\nobreak\discretionary{}%
{\usefont{OMS}{cmsy}{m}{n}\char50}{}}
\def\to{\mathchar"3221\nobreak\discretionary{}%
{\usefont{OMS}{cmsy}{m}{n}\char33}{}}
%%% Конец переносов математики %%%


%%% Запрет переносов заголовков %%%
% \makeatletter
% % запрещаем переносы в названиях секций
% \renewcommand{\section}{\@startsection{section}{1}{0pt}%
%                                 {-3.5ex plus -1ex minus -.2ex}%
%                                 {2.3ex plus .2ex}%
% {\flushleft\hyphenpenalty=10000\normalfont\Large\bfseries}}
% % меняем заголовок для команды \chapter и запрещаем переносы слов
% \usepackage {titlesec}
% \titleformat{\chapter}{\thispagestyle{myheadings}\flushleft\hyphenpenalty=10000\normalfont\huge\bfseries}{\thechapter\ }{0pt}{\Huge}
% \makeatother

%%% Оформление исходных текстов %%%
% \definecolor{mygreen}{rgb}{0,0.6,0}
% \definecolor{mygray}{rgb}{0.5,0.5,0.5}
% \definecolor{mymauve}{rgb}{0.58,0,0.82}
% \definecolor{deepred}{rgb}{0.6,0,0}

% \lstset{ %
%   backgroundcolor=\color{white},   % choose the background color; you must add \usepackage{color} or \usepackage{xcolor}
%   basicstyle=\scriptsize\ttfamily,        % the size of the fonts that are used for the code
%   breakatwhitespace=false,         % sets if automatic breaks should only happen at whitespace
%   breaklines=true,                 % sets automatic line breaking
%   captionpos=b,                    % sets the caption-position to bottom
%   commentstyle=\itshape\color{mygray},    % comment style
%   deletekeywords={...},            % if you want to delete keywords from the given language
%   % escapeinside={\%*}{*)},          % if you want to add LaTeX within your code
%   extendedchars=true,              % lets you use non-ASCII characters; for 8-bits encodings only, does not work with UTF-8
%   frame=L,                    % adds a frame around the code
%   % identifierstyle=\color{mymauve},
%   keepspaces=true,                 % keeps spaces in text, useful for keeping indentation of code (possibly needs columns=flexible)
%   keywordstyle=\bfseries\color{RoyalBlue},       % keyword style
%   % morekeywords={*,...},            % if you want to add more keywords to the set
%   numbers=left,                    % where to put the line-numbers; possible values are (none, left, right)
%   numbersep=10pt,                   % how far the line-numbers are from the code
%   numberstyle=\scriptsize\color{mygray}, % the style that is used for the line-numbers
%   rulecolor=\color{black},         % if not set, the frame-color may be changed on line-breaks within not-black text (e.g. comments (green here))
%   showspaces=false,                % show spaces everywhere adding particular underscores; it overrides 'showstringspaces'
%   showstringspaces=false,          % underline spaces within strings only
%   showtabs=false,                  % show tabs within strings adding particular underscores
%   stepnumber=1,                    % the step between two line-numbers. If it's 1, each line will be numbered
%   stringstyle=\bfseries\color{green!40!black},     % string literal style
%   tabsize=2,                       % sets default tabsize to 2 spaces
%   title=\lstname                   % show the filename of files included with \lstinputlisting; also try caption instead of title
% }

% \lstdefinestyle{perl}
% {
% 	language=Perl,
% 	alsoletter={\%},
% 	% basicstyle=\scriptsize\ttfamily,
% 	% keywordstyle=\bfseries\color{green!40!black},
% 	% commentstyle=\color{purple!40!black},
% 	% identifierstyle=\color{blue},
% 	emph={line},          % Custom highlighting
% 	emphstyle=\ttb\color{deepred},    % Custom highlighting style
% 	morecomment=[l][keywordstyle]{@\#},
% 	% stringstyle=\color{orange},
% }

% \lstdefinestyle{python}
% {
% 	language=Python,
% 	% alsoletter={...},
% 	% basicstyle=\scriptsize\ttfamily,
% 	% keywordstyle=\bfseries\color{green!40!black},
% 	% commentstyle=\color{purple!40!black},
% 	% identifierstyle=\color{blue},
% 	% morecomment={...},
% 	language=Python,
% 	otherkeywords={self},             % Add keywords here
% 	emph={local,resolver,props,newpid,context,desktop,url1,url2,document,dispatcher},          % Custom highlighting
% 	emphstyle=\ttb\color{deepred},    % Custom highlighting style
% 	showstringspaces=false            % 
% }

\renewcommand{\theFancyVerbLine}{
  \sffamily\textcolor[rgb]{0.5,0.5,0.5}{\scriptsize\arabic{FancyVerbLine}}}