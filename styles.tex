%%% Макет страницы %%%
\geometry{a4paper,top=2cm,bottom=2cm,left=2.5cm,right=1cm}
% \geometry{a4paper,top=2cm,bottom=2cm,left=3cm,right=1cm}

%%% Кодировки и шрифты %%%
\renewcommand{\rmdefault}{ftm}			% Включаем Times New Roman
% \renewcommand{\baselinestretch}{1.5}	% Полуторный интервал для всего текста
\onehalfspacing							% Полуторный интервал для всего текста

%%% Выравнивание и переносы %%%
\sloppy					% Избавляемся от переполнений
\clubpenalty=10000		% Запрещаем разрыв страницы после первой строки абзаца
\widowpenalty=10000		% Запрещаем разрыв страницы после последней строки абзаца
% \setlength{\parskip}{\baselineskip}

%%% Библиография %%%
\makeatletter
\bibliographystyle{utf8gost705u}	% Оформляем библиографию в соответствии с ГОСТ 7.0.5
\renewcommand{\@biblabel}[1]{#1}	% Заменяем библиографию с квадратных скобок на точку:
\makeatother

%%% Изображения %%%
\graphicspath{{./images/}} % Пути к изображениям

%%% Цвета гиперссылок %%%
\definecolor{linkcolor}{rgb}{0.0,0,0}
\definecolor{citecolor}{rgb}{0,0.0,0}
\definecolor{urlcolor}{rgb}{0,0,0}
\hypersetup{
    colorlinks, linkcolor={linkcolor},
    citecolor={citecolor}, urlcolor={urlcolor}
}

%%% Оглавление %%%
\renewcommand{\cftchapdotsep}{\cftdotsep}

%%% Переносы %%%
\begingroup
    \lccode`\==`\-
    % Пример: задать правила переноса для `коммерческо-посредническая':
    \hyphenation{бе-зо-пас-ность}
    \hyphenation{бе-зо-пас-ности}
    \hyphenation{бе-зо-пас-ностью}
    \hyphenation{ра-зом-кну-тую}
\endgroup
\newcommand{\ndash}{\nobreakdash--} % неразрывный перенос

%%% Подписи рисунков и таблиц %%%
\DeclareCaptionLabelSeparator{defffis}{ --- }
\captionsetup{justification=centering,labelsep=defffis}

%%% Тире в перечислениях %%%%
\renewcommand{\labelitemi}{\normalfont\bfseries{--}}

%%% Переносы математики %%%
\newcommand{\hm}[1]{#1\nobreak\discretionary{}{\hbox{\ensuremath{#1}}}{}}
\begingroup
\catcode`\+\active\gdef+{\mathchar8235\nobreak\discretionary{}%
{\usefont{OT1}{cmr}{m}{n}\char43}{}}
\catcode`\-\active\gdef-{\mathchar8704\nobreak\discretionary{}%
{\usefont{OMS}{cmsy}{m}{n}\char0}{}}
\catcode`\=\active\gdef={\mathchar12349\nobreak\discretionary{}%
{\usefont{OT1}{cmr}{m}{n}\char61}{}}
\catcode`\<\active\gdef<{\mathchar"313C\nobreak\discretionary{}%
{\usefont{OML}{cmm}{m}{n}\char60}{}}
\catcode`\>\active\gdef>{\mathchar"313E\nobreak\discretionary{}%
{\usefont{OML}{cmm}{m}{n}\char62}{}}
\endgroup
\def\times{\mathchar8706\nobreak\discretionary{}{\usefont{OMS}{cmsy}{m}{n}\char
2}{}}
\def\subset{\mathchar"321A\nobreak\discretionary{}%
{\usefont{OMS}{cmsy}{m}{n}\char26}{}}
%\supset,\subseteq,\notin
\def\neq{\not=\nobreak\discretionary{}%
{\usefont{OMS}{cmsy}{m}{n}\char54\usefont{OT1}{cmr}{m}{n}\char61}{}}
\def\sim{\mathchar"3218\nobreak\discretionary{}%
{\usefont{OMS}{cmsy}{m}{n}\char24}{}}
\def\in{\mathchar"3232\nobreak\discretionary{}%
{\usefont{OMS}{cmsy}{m}{n}\char50}{}}
\def\to{\mathchar"3221\nobreak\discretionary{}%
{\usefont{OMS}{cmsy}{m}{n}\char33}{}}
%%% Конец переносов математики %%%

%%% Оформление исходных текстов %%%
\renewcommand{\theFancyVerbLine}{
  \sffamily\textcolor[rgb]{0.5,0.5,0.5}{\scriptsize\arabic{FancyVerbLine}}}