\section{Выбор системы контроля версий} \label{technologic_cvs}

Для хранения истории вносимых изменений, а также синхронизации данных между хранилищами будет использоваться система контроля версий --- программное обеспечение, разработанное для удобства совместной разработки ПО: синхронизации данных и слияния веток разработки. Такие системы решают задачи, схожие с задачами  электронного документооборота, и на их основе можно строить СЭД \cite{my_conference_2013}.

\vspace{\baselineskip}
Системы контроля версий делятся на:
\begin{itemize}
		\item Централизованные --- имеется общий сервер для хранения документов, клиенты работают в режиме он-лайн;
		\item Децентрализованные --- все участники информационного обмена равноправны. Обычно на уровне политики принимается некоторый центральный сервер, который считается эталоном. Пользователи могут работать в режиме оффлайн, время от времени синхронизируя данные и сливая ветки разработки друг с другом.
	\end{itemize}

\vspace{\baselineskip}
В применении к системе электронного документооборота разумным выглядит применение децентрализованной системы ввиду масштабируемости и гибкости развёртывания, а также простоте создания резервных копий. Среди всех децентрализованных систем контроля версий была выбрана система Git как самая распространённая и хорошо документированная.