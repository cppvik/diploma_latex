\section{Выбор системы контроля версий} \label{technologic_cvs}

Для хранения истории вносимых изменений, а также синхронизации данных между хранилищами будет использоваться система контроля версий --- программное обеспечение, разработанное для удобства совместной разработки ПО: синхронизации данных и слияния веток разработки. Такие системы решают задачи, схожие с задачами  электронного документооборота, и на их основе можно строить СЭД \cite{my_conference_2013}.

\vspace{\baselineskip}
Системы контроля версий делятся на:
\begin{itemize}
		\item Централизованные --- имеется общий сервер для хранения документов, клиенты работают в режиме <<онлайн>>;
		\item Децентрализованные --- все участники информационного обмена равноправны. Обычно на административном уровне принимается некоторый центральный сервер, который считается эталоном. Пользователи могут работать в режиме <<оффлайн>>, время от времени синхронизируя данные и сливая ветки разработки друг с другом.
	\end{itemize}

\vspace{\baselineskip}
В применении к системе электронного документооборота разумным выглядит применение децентрализованной системы (см. \ref{technologic_network}) ввиду масштабируемости и гибкости развёртывания, а также простоте создания резервных копий. 

\vspace{\baselineskip}
\textbf{\textit{Git}} --- распределённая система управления версиями файлов. Проект был создан Линусом Торвальдсом для управления разработкой ядра Linux, первая версия выпущена 7 апреля 2005 года.

Примерами проектов, использующих Git, являются ядро Linux, Android, Drupal, GNU Core Utilities, Wine, Chromium, Compiz Fusion, jQuery, PHP, MediaWiki и некоторые дистрибутивы Linux.

Программа является свободной и выпущена под лицензией GNU GPL версии 2.

\vspace{\baselineskip}
\textbf{\textit{Mercurial}} --- кроссплатформенная распределенная система управления версиями, разработанная для эффективной работы с очень большими репозиториями кода. 

Примерами проектов, использующих Mercurial, являются Mozilla, OpenOffice, Vim, Gajim, Go, CPython, NetBeans.

Программа является свободной и выпущена под лицензией GNU GPL v2.

\vspace{\baselineskip}
Для дальнейшей разработки была выбрана система Git . Она эффективно работает с бинарными файлами, что актуально для СЭД, а также является широко распространённой и хорошо документированной \cite{my_conference_2013}.