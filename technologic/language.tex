\section{Выбор языка реализации} \label{technologic_langiage}


При выборе языка реализации, разработчик сталкивается с необходимостью учёта следующих аспектов:
\begin{itemize}
	\item возможность разработки приложений, поддерживаемых большинством современных веб\ndashбраузеров;
	\item возможность разработки приложений, поддерживаемых большинством современных клиентских платформ;
	\item особенности системы разработки программного обеспечения (простота использования, открытость, наличие доступной справочной документации);
	\item опыт конкретного разработчика в области проектирования и написания кода для этого языка программирования.
\end{itemize}

С учётом перечисленных требований, в качестве возможных языков программирования рассматривались: Perl, Bash, Python, PHP.

\vspace{\baselineskip}
\textbf{\textit{Perl}} ---  высокоуровневый интерпретируемый динамический язык программирования общего назначения. Название языка представляет собой аббревиатуру, которая расшифровывается как Practical Extraction and Report Language — <<практический язык для извлечения данных и составления отчётов>>.

Основной особенностью языка считаются его богатые возможности для работы с текстом, в том числе работа с регулярными выражениями, встроенная в синтаксис. Perl унаследовал многие свойств языков C, AWK, а также скриптовых языков командных оболочек UNIX.

Язык программирования Perl обладает большой коллекцией дополнительных модулей CPAN \cite{cpan}.

\vspace{\baselineskip}
\textbf{\textit{Bash}} --- одна из наиболее популярных современных разновидностей командной оболочки UNIX. Она широко распространена в среде Linux, где часто используется в качестве предустановленной командной оболочки.

Bash --- это командный процессор, работающий, как правило, в интерактивном режиме в текстовом окне. Bash также может читать команды из файла, который называется скриптом (или сценарием). Как и все Unix\ndashоболочки, он поддерживает автодополнение названий файлов и папок, подстановку вывода результата команд, переменные, контроль за порядком выполнения, операторы ветвления и цикла.

\vspace{\baselineskip}
\textbf{\textit{Python}} ---  высокоуровневый язык программирования общего назначения, ориентированный на повышение производительности разработчика и читаемости кода. Синтаксис ядра Python минималистичен, в то же время стандартная библиотека включает большой объём полезных функций.

Python поддерживает несколько парадигм программирования, в том числе структурное, объектно\ndashориентированное, функциональное, императивное и аспектно\ndashориентированное. Основные архитектурные черты --- динамическая типизация, автоматическое управление памятью, полная интроспекция, механизм обработки исключений, поддержка многопоточных вычислений и удобные высокоуровневые структуры данных. Код в Python организовывается в функции и классы, которые могут объединяться в модули (они, в свою очередь, могут быть объединены в пакеты).

Python --- активно развивающийся язык программирования, новые версии выходят примерно раз в два с половиной года.

\vspace{\baselineskip}
\textbf{\textit{PHP}} ---  скриптовый язык программирования общего назначения, интенсивно применяемый для разработки веб\ndashприложений. В настоящее время поддерживается подавляющим большинством хостинг\ndashпровайдеров и является одним из лидеров среди языков программирования, применяющихся для создания динамических веб\ndashсайтов.

Проект распространяется под собственной лицензией, несовместимой с GNU GPL.

\vspace{\baselineskip}
Все перечисленные языки относятся к скриптовым, но предназначены в общем случае для выполнения разных задач. Однако, например, для веб\ndashразработки кроме PHP может использоваться и Perl.

\vspace{\baselineskip}
С учётом имеющихся наработок в области программирования пользовательских сценариев для работы в системах контроля версий было отдано предпочтение языкам Perl, Bash и Python.

\vspace{\baselineskip}
Для конфигурации поведенческих сценариев будет использоваться язык Bash, а для модификации средства просмотра истории gitweb --- Perl. Средства интеграции с офисными пакетами Microsoft Office и LibreOffice будут написаны на языке Python, который является наиболее удобным языком с точки зрения LibreOffice API. Так как данные языки являются открытыми и свободно распространяемыми, а также не требуют специальных средств отладки, в качестве сред исполнения будут выступать стандартные интерпретаторы Perl, Python и Bash, а также PowerShell для Windows.