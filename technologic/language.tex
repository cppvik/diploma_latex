\section{Выбор языка реализации} \label{technologic_langiage}


При выборе языка реализации, разработчик сталкивается с необходимостью учета следующих аспектов:
\begin{itemize}
	\item возможность разработки приложений, поддерживаемых большинством современных веб-браузеров;
	\item возможность разработки приложений, поддерживаемых большинством современных клиентских платформ;
	\item особенности системы разработки программного обеспечения (простота использования, открытость, наличие доступной справочной документации);
	\item опыт конкретного разработчика в области проектирования и написания кода для этого языка программирования.
\end{itemize}

С учётом перечисленных требований, в качестве возможных языков программирования рассматривались: Perl, Bash, Python, PHP. Все они являются гибкими и современными и позволяют решить поставленную задачу. С учётом имеющихся наработок в области программирования пользовательских сценариев для работы в системах контроля версий было отдано предпочтение языкам Perl и Bash.

\vspace{\baselineskip}
Для конфигурации поведенческих сценариев будет использоваться язык Bash, а для модификации средства просмотра истории gitweb --- Perl. Так как данные языки являются открытыми и свободно распространяемыми, а также не требуют специальных средств отладки, в качестве среды исполнения будут выступать стандартные интерпретаторы Perl и Bash, а также PowerShell для Windows.