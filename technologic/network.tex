\section{Организация сетевого взаимодействия} \label{technologic_network}

В качестве модели сетевого взаимодействия возможно использование двух подходов: централизованного и децентрализованного.

\vspace{\baselineskip}
\textit{Централизованная} модель подразумевает наличие нескольких клиентов и одного сервера, координирующего их работу. К этому серверу предъявляются повышенные требования надёжности и отказоустойчивости, так как его отказ приводит к нарушению функционирования всей системы.

\vspace{\baselineskip}
\textit{Децентрализованный} подход подразумевает создание более гибкой сети. Возможно выстраивание иерархии серверов, каждый из которых обслуживает некий (возможно уникальный) набор клиентов. При таком подходе сервера низших звеньев могут являться клиентами более крупных серверов, что позволяет обеспечить более простое и удобное резервирование сервисов и данных.