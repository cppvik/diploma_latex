\section{Защита канала передачи данных} \label{technologic_network_security}

При организации сетевого взаимодействия важным становится вопрос защиты канала передачи данных.
Как описано в разделе (\ref{threats_neutralizers}), для аутентификации клиентов лучше всего использовать не пароли, а открытые ключи пользователей, причём последние могут храниться на аппаратных ключевых носителях.
Данный метод реализован в протоколах HTTPS и SSH.
Оба протокола достаточно распространены и доступны во всех современных операционных системах.

\vspace{\baselineskip}
В разрабатываемом программном обеспечении реализованы оба протокола с механизмом двусторонней аутентификации, в том числе с использованием аппаратных ключевых носителей.
Выбор протокола для конкретной системы осуществляется с учётом её области применения.
Так, если приоритетным является доступ на чтение, а также работа со средствами удалённого просмотра, целесообразно использование HTTPS.
SSH полезен при организации децентрализованной распределённой сети с акцентом на запись изменений.

\vspace{\baselineskip}
Возможно также использование двух протоколов одновременно: предоставление доступа на чтение по HTTPS и полноценное функционирование с помощью SSH.